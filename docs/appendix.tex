%%%%%%%%%% APPENDIX %%%%%%%%%%
\section{\label{sec:chnxt_api}ChNXT Class API}
The header file nxt.h defines all the data types, macros and function prototypes 
for th Lego Mindstorms NXT API library. The header file declares a class called 
ChNXT which contains member functions which may be used to control the Lego
Minstorms NXT.\\
%% DATA TYPE %%
\subsection{\label{sec:datatypes}Data Types Used in ChNXT Class}
The data types defined in the header file {\tt nxt.h} are described in this 
appendix. These data types are used by the NXT library to represent certain values, 
such as joint id's and motor directions.\\

\noindent
\begin{longtable}{p{3.5cm}p{12cm}} 
    \hline
    Data Type& Description \\
    \hline 
    {\tt nxtJointId\_t} & An enumerated value that indicates a 
    nxt joints. \\
    {\tt nxtJointState\_t} & The current state of a nxt joint. \\
    {\tt nxtSensorId\_t} & An enumerate value that indicates a 
    nxt's sensors. \\
    {\tt nxtSensorType\_t} & An enumerate value that indicates 
    the type of a sensor of a nxt.\\
    {\tt nxtSensorMode\_t} & An enumerate value that indicates 
    the mode of a sensor for getting value. \\
    \hline
\end{longtable}

\subsubsection{\label{sec:nxtJointId_t}{\tt nxtJointId\_t}}
This data type is an enumerated type used to identify a joint on 
the Lego Mindstorms NXT. Valid values for this type are:
\begin{lstlisting}
typedef enum nxt_joints_e {
  NXT_JOINT1 = 0,
  NXT_JOINT2 = 1,
  NXT_JOINT3 = 2,
} nxtJointId_t;
\end{lstlisting}
\index{nxt\_joints\_t}
\index{NXT\_JOINT1}
\index{NXT\_JOINT2}
\index{NXT\_JOINT3}

\noindent
\begin{longtable}{p{3.5cm}p{12cm}} 
    \hline 
    Value & Description \\
    \hline 
    {\tt NXT\_JOINT1} & PortA on the Lego Mindstorms NXT. \\
    {\tt NXT\_JOINT2} & PortB on the Lego Mindstorms NXT. \\
    {\tt NXT\_JOINT3} & PortC on the Lego Mindstorms NXT. \\
    \hline
\end{longtable}

\subsubsection{\label{sec:nxtJointState_t}{\tt nxtJointState\_t}}
This datatype is an enumerated type used to designate the current 
movement state of a joint. The values may be retrieved from the 
robot with the {\tt getJointState()} function and may be set 
with the {\tt moveContinuous()} family of functions. Valid values are:
\begin{lstlisting}
typedef enum nxt_joint_state_e {
    NXT_FORWARD   = 1,
    NXT_BACKWARD  = -1,
} nxtJointState_t;
\end{lstlisting}
\index{nxtJointState\_t}
\index{NXT\_FORWARD}
\index{NXT\_BACKWARD}

\noindent
\begin{longtable}{p{3.5cm}p{12cm}}
    \hline
    Value & Description \\
    \hline
    {\tt NXT\_FORWARD} & This value indicates that the joint is currently moving forward.  \\
{\tt NXT\_BACKWARD}& This value indicates that the joint is currently moving backward. \\
    \hline
\end{longtable}

\subsubsection{\label{sec:nxtSensorPort_t}{\tt nxtSensorPort\_t}}
This datatype is an enumerated type used to identify a sensor on 
the Lego Mindstorms NXT. Valid values for this type are:
\begin{lstlisting}
typedef enum nxt_sensors_e{
    NXT_SENSORPORT1 = 0,
    NXT_SENSORPORT2 = 1,
    NXT_SENSORPORT3 = 2,
    NXT_SENSORPORT4 = 3
} nxtSensorId_t;
\end{lstlisting}
\index{nxtSensorId\_t}
\index{NXT\_SENSORPORT1}
\index{NXT\_SENSORPORT2}
\index{NXT\_SENSORPORT3}
\index{NXT\_SENSORPORT4}

\noindent
\begin{longtable}{p{3.5cm}p{12cm}}
    \hline
    Value &       Description\\
    \hline
    {\tt NXT\_SENSORPORT1}&Select sensor input PORT 1 on the NXT.\\
    {\tt NXT\_SENSORPORT2}&Select sensor input PORT 2 on the NXT.\\
    {\tt NXT\_SENSORPORT3}&Select sensor input PORT 3 on the NXT.\\
    {\tt NXT\_SENSORPORT4}&Select sensor input PORT 4 on the NXT.\\
    \hline
\end{longtable}

\subsubsection{\label{sec:nxtSensorType_t}{\tt nxtSensorType\_t}}
This data type is used to identify the type of a sensor for the Lego Mindstorms 
NXT. Valid values for this type are:
\begin{lstlisting}
typedef enum nxt_sensor_type_e{
    NXT_SENSORTYPE_SWITCH          = 0x01,
    NXT_SENSORTYPE_TEMPERATURE     = 0x02,
    NXT_SENSORTYPE_LIGHT_ACTIVE    = 0x05,
    NXT_SENSORTYPE_LIGHT_INACTIVE  = 0x06,
    NXT_SENSORTYPE_SOUND_DB        = 0x07,
    NXT_SENSORTYPE_SOUND_DBA       = 0x08,
    NXT_SENSORTYPE_LOWSPEED        = 0x0A,
    NXT_SENSORTYPE_9V              = 0x0B,
    NXT_SENSORTYPE_HIGHSPEED       = 0x0C, /* No useage now */
    NXT_SENSORTYPE_COLORFULL       = 0x0D,
    NXT_SENSORTYPE_COLORRED        = 0x0E,
    NXT_SENSORTYPE_COLORGREEN      = 0x0F,
    NXT_SENSORTYPE_COLORBLUE       = 0x10,
    NXT_SENSORTYPE_COLORNONE       = 0x11
} nxtSensorType_t;

#define NXT_SENSORTYPE_TOUCH      NXT_SENSORTYPE_SWITCH
#define NXT_SENSORTYPE_ULTRASONIC NXT_SENSORTYPE_LOWSPEED_9V
\end{lstlisting}
\index{nxtSensorType\_t}
\index{NXT\_SENSORTYPE\_SWITCH}
\index{NXT\_SENSORTYPE\_TEMPERATURE}
\index{NXT\_SENSORTYPE\_LIGHT\_ACTIVE}
\index{NXT\_SENSORTYPE\_LIGHT\_INACTIVE}
\index{NXT\_SENSORTYPE\_SOUND\_DB}
\index{NXT\_SENSORTYPE\_SOUND\_DBA}
\index{NXT\_SENSORTYPE\_LOWSPEED}
\index{NXT\_SENSORTYPE\_LOWSPEED\_9V}
\index{NXT\_SENSORTYPE\_HIGHTSPEED}
\index{NXT\_SENSORTYPE\_COLORFULL}
\index{NXT\_SENSORTYPE\_COLORRED}
\index{NXT\_SENSORTYPE\_COLORGREEN}
\index{NXT\_SENSORTYPE\_COLORBLUE}
\index{NXT\_SENSORTYPE\_COLORNONE}
\index{NXT\_SENSORTYPE\_TOUCH}
\index{NXT\_SENSORTYPE\_ULTRASONIC}

\noindent
\begin{longtable}{p{5.5cm}p{10cm}}
    \hline
    Value &       Description\\
    \hline
    {\tt NXT\_SENSORTYPE\_SWITCH}        &Set to a switch type sensor. Touch 
    sensor is a switch type sensor.\\
    {\tt NXT\_SENSORTYPE\_TEMPERATURE}   &Set to Temperature Sensor.\\
    {\tt NXT\_SENSORTYPE\_LIGHT\_ACTIVE} &Set to Light Sensor in light active 
    mode(LED on).\\
    {\tt NXT\_SENSORTYPE\_LIGHT\_INACTIVE}&Set to Light Sensor in light inactive 
    mode(LED off).\\
    {\tt NXT\_SENSORTYPE\_SOUND\_DB}    &Set to Sound Sensor in dB.\\
    {\tt NXT\_SENSORTYPE\_SOUND\_DBA}   &Set to Sound Sensor in dB with adjusted.\\
    {\tt NXT\_SENSORTYPE\_LOWSPEED}     &Set to ISP type sensor.\\
    {\tt NXT\_SENSORTYPE\_LOWSPEED\_9V} &Set to ISP type sensor with 9 Voltage. 
    The ultrasonic sensor belongs to this type of sensor.\\
    {\tt NXT\_SENSORTYPE\_HIGHSPEED}     &Not avialable now.\\
    {\tt NXT\_SENSORTYPE\_COLORFULL}     &Set to Color Sensor in color detector mode.\\
    {\tt NXT\_SENSORTYPE\_COLORRED}      &Set to Color Sensor in lightsensor mode 
    with red light on.\\
    {\tt NXT\_SENSORTYPE\_COLORGREEN}    &Set to Color Sensor in lightsensor mode 
    with green light on.\\
    {\tt NXT\_SENSORTYPE\_COLORBLUE}     &Set to Color Sensor in lightsensor 
    mode with blue light on.\\
    {\tt NXT\_SENSORTYPE\_COLORNONE}     &Set to Color Sensor in lightsensor 
    mode with no light on.\\
    \hline
\end{longtable}
%\end{tabular}

\subsubsection{\label{sec:nxtSensorMode_t}{\tt nxtSensorMode\_t}}
This data type is used to identify the mode for a sensor to get value for the 
Lego Mindstorms NXT. Valid values for this type are:
\begin{lstlisting}
typedef enum nxt_sensor_mode_e{
    NXT_SENSORMODE_RAWMODE           = 0x00,
    NXT_SENSORMODE_BOOLEANMODE       = 0x20
    NXT_SENSORMODE_TRANSITIONCNTMODE = 0x40,
    NXT_SENSORMODE_PERIODCOUNTERMODE = 0x60,
    NXT_SENSORMODE_PCTFULLSCALEMODE  = 0x80,
    NXT_SENSORMODE_CELSIUSMODE       = 0xA0,
    NXT_SENSORMODE_FAHRENHEITMODE    = 0xC0
} nxtSensorMode_t;
\end{lstlisting}
\index{nxtSensorMode\_t}
\index{NXT\_SENSORMODE\_RAWMODE}
\index{NXT\_SENSORMODE\_BOOLEANMODE}
\index{NXT\_SENSORMODE\_TRANSITIONCNTMODE}
\index{NXT\_SENSORMODE\_PERIODCOUNTERMODE}
\index{NXT\_SENSORMODE\_PCTFULLSCALEMODE}
\index{NXT\_SENSORMODE\_CELSIUSMODE}
\index{NXT\_SENSORMODE\_FAHRENHEITMODE}

\noindent
\begin{longtable}{p{6cm}p{9.5cm}} 
    \hline
    Value & Description\\
    \hline
    {\tt NXT\_SENSORMODE\_RAWMODE}      &Get sensor value as raw mode.\\	 
    {\tt NXT\_SENSORMODE\_BOOLEANMODE}  &Get sensor value as boolean mode.\\
    {\tt NXT\_SENSORMODE\_TRANSITIONCNTMODE}  &Get sensor value as a number of 
    transitions between TRUE and FALSE.\\
    {\tt NXT\_SENSORMODE\_COUNTERMODE}  &Get sensor value as a number of 
    transitions from FALSE to TRUE, then back to FALSE.\\
    {\tt NXT\_SENSORMODE\_PCTFULLSCALEDMODE}  &Get sensor value as percentage of 
    full scale reading for configured sensor type.\\
    {\tt NXT\_SENSORMODE\_CELSIUSMODE}  &Get sensor value of temperature in 
    degrees Celsius.\\
    {\tt NXT\_SENSORMODE\_FAHRENHEITMODE}  &Get sensor value of temperature in 
    degrees Fahrenheit.\\
    \hline
\end{longtable}
%%%%%%%%%% End of Data Types %%%%%%%%%%

%%%%%%%%%%%%%%%%%%%%%%%%%%% ChNXT Class API %%%%%%%%%%%%%%%%%%%%%%%%%%%%

%%%%%%%%%%%%%%%% the table of communication functions %%%%%%%%%%%%
\subsection{ChNXT Class API Overview}
%\begin{table}[H]
\begin{longtable}{p{6cm}p{10cm}}
\caption{Functions for Communication}\\
\hline
Function & Description\\
\hline
{\tt connect()}&Connects to the Mindstorms NXT using bluetooth 
(Read bluetooth address from the configuration file automaticly).\\
{\tt connectWithAddress()}&Connects to the Mindstorms NXT via 
bluetooth (Read bluetooth address from users' input).\\
{\tt disconnect()}  &Disconnects from the Mindstorms NXT.\\
\hline
\end{longtable}
%\end{table}
% the table for the checking functions %
%\begin{table}[H]
\begin{longtable}{ p{6cm}p{10cm}}
\caption{Functions for Checking Status}\\
\hline
Function & Description\\
\hline
{\tt isConnected()}    &Check the connection to a robot.\\
{\tt isMoving()}       &Check if any joints are currently in motion.\\
\hline
\end{longtable}
%\end{table}
% the table for the sensors' functions %
%\begin{table}[H]
\begin{longtable}{p{6cm}p{10cm}}
\caption{Functions for Sensors}\\
\hline
Function & Description\\
\hline
{\tt setSensor()}       &Setup the sensors to collect data from the environment.\\
{\tt getSensor()}       &Get the data collected by the sensors from the NXT.\\
\hline
\end{longtable}
%\end{table}

%%%%%%%%%%%%%%% functions for joints %%%%%%%%%%%%%%
%\begin{table}[H]
\begin{longtable}{p{6cm}p{10cm}}
\caption{Functions for Joints}\\
\hline
%\begin{tabular}{ p{6cm}p{10cm} }\hline
Functions & Description\\
\hline
{\tt setJointZero()}         &Set the tachometer to zero for a single joint.\\
{\tt setToZero()}            &Set the tachometer to zero for all joints.\\
{\tt setJointSpeedRatio()}   &Set up a single joint speed.\\
{\tt setJointSpeedRatios()}  &Set up all joints' speed.\\
{\tt moveJointContinuousNB()}&Make a single joint move continuously.\\
{\tt moveJoint()}            &Make a single joint move a user-specified angle.\\
{\tt moveJointNB()}          &Identical to {\tt moveJoint()} but non-blocking.\\
{\tt moveJointTo()}          &Make a single joint move to an absolute angle.\\
{\tt moveJointToNB()}        &Identical to {\tt moveJointTo()} but non-blocking.\\
{\tt move()}                 &Make all joints move a user-specified angles.\\
{\tt moveNB()}               &Identical to {\tt move()} but non-blocking.\\
{\tt moveTo()}               &Make all joints move to absolute angles.\\
{\tt moveToNB()}             &Identical to {\tt moveTo()} but non-blocking.\\
{\tt moveToZero()}           &Make all joints move to absolute zero positions.\\
{\tt moveToZeroNB()}         &Identical to {\tt moveToZero()} but non-blocking.\\
{\tt moveContinuousNB()}     &Make all joints move continuously.\\
{\tt moveContinuousTime()}   &Make all joints move continuously for a certain amount of time.\\
{\tt moveWait()}             &Wait untill all motors have stopped moving.\\
{\tt getJointAngle()}        &Get tachometer counts from NXT.\\
{\tt getJointSpeedRatio()}   &Get a motor's speed ratio from NXT.\\
{\tt getJointSpeedRatios()}  &Get all motors' speed ratios from NXT.\\
{\tt stopOneJoint()}         &Make a single joint stop moving.\\
{\tt stopTwoJoints()}        &Make two joints stop moving.\\
{\tt stopAllJoints()}        &Make all joints stop moving.\\
\hline
\end{longtable}
%\end{tabular}
%\end{table}

% the functions for vehicle configuration %
%\begin{table}[H]
\begin{longtable}{p{6cm}p{10cm}}
\caption{Functions for the Vehicle Configuration
\label{tab:function_vehicle}}\\
\hline
Functions & Description\\
\hline
{\tt vehicleRollForward()}      &Moves the NXT vehicle forward.\\
{\tt vehicleRollForwardNB()}    &Identical to {\tt vehicleRollForward()}, but non-blocking.\\
{\tt vehicleRollBackward()}     &Moves the NXT vehicle backward.\\
{\tt vehicleRollBackwardNB()}   &Identical to {\tt vehicleRollBackward()}, but non-blocking.\\
{\tt vehicleRotateLeft()}       &Rotates the NXT vehicle left.\\
{\tt vehicleRotateLeftNB()}     &Identical to {\tt vehicleRotateLeft()}, but non-blocking.\\
{\tt vehicleRotateRight()}      &Rotates the NXT vehicle right.\\
{\tt vehicleRotateRightNB()}    &Identical to {\tt vehicleRotateRight()}, but non-blocking.\\
{\tt vehicleMotionWait()}       &Wait until vehicle stop moving.\\
\hline
\end{longtable}
%\end{table}

%% the functions for humanoid configuration %%
%\begin{table}[H]
\begin{longtable}{p{6cm}p{10cm}}
    \caption{Functions for the Humanoid Configuration
    \label{tab:function_humanoid}}\\
    \hline
    Functions & Description\\ \hline
        {\tt humanoidWalkForward()}   &Moves the NXT humanoid forward.\\
        {\tt humanoidWalkForwardNB()} &Identical to {\tt humanoidWalkForward()}, but non-blocking.\\
        {\tt humanoidWalkBackward()}  &Moves the NXT humanoid backward.\\
        {\tt humanoidWalkForwardNB()} &Identical to {\tt humanoidWalkBackward()}, but non-blocking.\\
        {\tt humanoidMotionWait()}    &Wait until humanoid robot stop moving.\\
    \hline
\end{longtable}
%\end{table}

\clearpage
\newpage
% functions details %
\subsection{ChNXT Class API Details}
\noindent
\vspace{5pt}
\rule{4.5in}{0.015in}\\
\noindent
{\LARGE \texttt{ChNXT::connect()} \index{ChNXT::connect()}}\\

\addcontentsline{toc}{subsection}{connect()}

\noindent
{\bf Synopsis}
\vspace{-8pt}
\begin{verbatim}
#include <nxt.h>
int ChNXT::connect();
\end{verbatim}

\noindent
{\bf Purpose}\\
Connect to a Mindstorms NXT via Bluetooth.\\

\noindent
{\bf Return Value}\\
The function returns 0 on success, and non-zero otherwise.\\

\noindent
{\bf Parameters}\\
None.\\

\noindent
{\bf Description}\\
This function is used to connect to a Mindstorms NXT via Bluetooth. The Bluetooth address is gotten from the configuration file.\\

\noindent
{\bf Example}
\begin{verbatim}
#include <nxt.h> 
#include <stdio.h>

ChNXT nxt;
if (nxt.connect()){
    printf("Connection to NXT has failed.");
    exit(0);
}
    
nxt.disconnect();
\end{verbatim}

\noindent
{\bf See Also}\\
\texttt{connectWithAddress()}, \texttt{disconect()}\\
%%%%% END OF ChNXT::connect %%%%%

\noindent
\vspace{5pt}
\rule{4.5in}{0.015in}\\
\noindent
{\LARGE \texttt{ChNXT::connectWithAddress()} \index{ChNXT::connectWithAddress()}}\\

\addcontentsline{toc}{subsection}{connectWithAddress()}

\noindent
{\bf Synopsis}
\vspace{-8pt}
\begin{verbatim}
#include <nxt.h>
int ChNXT::connectWithAddress(char usr_addr[18]);
\end{verbatim}

\noindent
{\bf Purpose}\\
Connect to a Mindstorms NXT via Bluetooth.\\

\noindent
{\bf Return Value}\\
The function returns 0 on success, and non-zero otherwise.\\

\noindent
{\bf Parameters}\\
\vspace{-0.1in}
\begin{description}
\item
\begin{tabular}{p{20mm}p{135mm}}
\texttt{usr\_address} & The Bluetooth address of the Mindstorm NXT.
\end{tabular}
\end{description}

\noindent
{\bf Description}\\
This function is used to connect to a Mindstorms NXT via Bluetooth. The Bluetooth address is gotten from the configuration file.\\

\noindent
{\bf Example}
\begin{verbatim}
#include <nxt.h> 

ChNXT nxt;
if (nxt.connectWithAddress("00:16:53:12:e7:80")){
    printf("Connection to NXT has failed.");
    exit(0);
}
\end{verbatim}

\noindent
{\bf See Also}\\
\texttt{connect()}, \texttt{disconnect()}\\
%%%%% END OF ChNXT::connectWithAddress %%%%%

\noindent
\vspace{5pt}
\rule{4.5in}{0.015in}\\
\noindent
{\LARGE \texttt{ChNXT::disconnect()} \index{ChNXT::disconnect()}}\\

\addcontentsline{toc}{subsection}{disconnect()}

\noindent
{\bf Synopsis}
\begin{lstlisting}
#include <nxt.h>
int ChNXT::disconnect();
\end{lstlisting}

\noindent
{\bf Purpose}\\
Disconnect from a Lego mindstorms NXT.\\

\noindent
{\bf Return Value}\\
The function returns 0 on success and non-zero otherwise.\\

\noindent
{\bf Parameters}\\
None.\\

\noindent
{\bf Description}\\
This function is used to disconnect from a connected Mindstorms NXT. A call to this function is not necessary before the terminationof a program. It is only neccessary if another connection will be established within the same program at a later time.\\

\noindent
{\bf Example}
\begin{lstlisting}
#include <nxt.h> 

ChNXT nxt;
if (nxt.connectWithAddress("00:16:53:12:e7:80")){
    printf("Connection to NXT has failed.");
    exit(0);
}
    
nxt.disconnect();
\end{lstlisting}

\noindent
{\bf See Also}\\
\texttt{connect()}, \texttt{connectWithAddress}\\
%%%%% END OF ChNXT::connect %%%%%

%\newpage
\noindent
\vspace{5pt}
\rule{4.5in}{0.015in}\\
\noindent
{\LARGE \texttt{ChNXT::getMotorAngle()}\index{ChNXT::getMotorAngle()}}\\
%\phantomsection
\addcontentsline{toc}{subsection}{getMotorAngle()}

\noindent
{\bf Synopsis}
\begin{lstlisting}
#include <nxt.h>
int ChNXT::getMotorAngle(nxtMotorId_t id, double &angle);
\end{lstlisting}

\noindent
{\bf Purpose}\\
Retrieve a robot motor's current angle.\\

\noindent
{\bf Return Value}\\
The function returns 0 on success and non-zero otherwise.\\

\noindent
{\bf Parameters}\\
\vspace{-0.1in}
\begin{description}
\item               
\begin{tabular}{p{15 mm}p{145 mm}}
\texttt{id} & The motor number. This is an enumerated type 
discussed in Section \ref{sec:nxtMotorId_t} on page
\pageref{sec:nxtMotorId_t}.\\
\texttt{angle} & A variable to store the current angle of the robot
motor. The contents of this variable will be overwritten with a value that
represents the motor's angle in degrees.  \\
\end{tabular}
\end{description}

\noindent
{\bf Description}\\
This function gets the current motor angle of a NXT's motor. The
angle returned is in units of degrees and is accurate to roughly $\pm1$
degrees. The function \texttt{getMotorAngle()} always returns an angle 
from 0 to $\infty$ degrees.\\

\noindent
{\bf Example}
\begin{lstlisting}
#include <nxt.h>

ChNXT nxt;
double angle;

nxt.connect();

nxt.getMotorAngle(NXT_MOTORA, angle);

printf("The angle of motorA is: %lf\n", angle);
\end{lstlisting}

\noindent
{\bf See Also}\\

%\CPlot::\DataThreeD(), \CPlot::\DataFile(), \CPlot::\Plotting(), \plotxy().\\

%\noindent
\vspace{5pt}
\rule{4.5in}{0.015in}\\
\noindent
{\LARGE \texttt{ChNXT::getMotorDirection()}\index{ChNXT::getMotorDirection()}}\\
%\phantomsection
\addcontentsline{toc}{subsection}{getMotorDirection()}

\noindent
{\bf Synopsis}
\begin{lstlisting}
#include <nxt.h>
int ChNXT::getMotorDirection(nxtMotorId_t id, int &direction);
\end{lstlisting}

\noindent
{\bf Purpose}\\
Get the speed of a motor on the Lego Mindstorms NXT.\\

\noindent
{\bf Return Value}\\
The function returns 0 on success and non-zero otherwise.\\

\noindent
{\bf Parameters}
\vspace{-0.1in}
\begin{description}
\item               
\begin{tabular}{p{10 mm}p{145 mm}}
\texttt{id} & The motor number to pose. This is an enumerated type 
discussed in Section \ref{sec:robotMotorId_t} on page
\pageref{sec:robotMotorId_t}.\\
\texttt{direction} & An integer variable. This variable will be overwritten
with the current speed of the motor.
\end{tabular}
\end{description}

\noindent
{\bf Description}\\
This function is used to retrieve the motor's direction status. The valid
status directions are
\begin{itemize}
\item 1: Forward direction
\item 2: Backward direction
\end{itemize}

\noindent
{\bf Example}
\noindent

\noindent
{\bf See Also}\\
\texttt{setMotorDirection()}

%\CPlot::\DataThreeD(), \CPlot::\DataFile(), \CPlot::\Plotting(), \plotxy().\\

%\newpage
\noindent
\vspace{5pt}
\rule{4.5in}{0.015in}\\
\noindent
{\LARGE \texttt{ChNXT::getMotorSpeedRatio()}\index{ChNXT::getMotorSpeedRatio()}}\\
%\phantomsection
\addcontentsline{toc}{subsection}{getMotorSpeedRatio()}

\noindent
{\bf Synopsis}
\begin{lstlisting}
#include <nxt.h>
int ChNXT::getMotorSpeedRatio(nxtMotorId_t id, double &ratio);
\end{lstlisting}

\noindent
{\bf Purpose}\\
Get the speed ratio settings of a motor on the Lego Mindstorms NXT.\\

\noindent
{\bf Return Value}\\
The function returns 0 on success and non-zero otherwise.\\

\noindent
{\bf Parameters}
\vspace{-0.1in}
\begin{description}
\item               
\begin{tabular}{p{10 mm}p{145 mm}}
\texttt{id} & Retrieve the speed ratio setting of this motor. This is an 
enumerated type discussed in Section \ref{sec:nxtMotorId_t} on page
\pageref{sec:nxtMotorId_t}.\\
\texttt{ratio} & A variable of type double. The value of this variable will
be overwritten with the current speed ratio setting of the motor.
\end{tabular}
\end{description}

\noindent
{\bf Description}\\
This function is used to find the speed ratio setting of a motor. The speed
ratio setting of a motor is the percentage of the maximum motor speed, and the
value ranges from 0 to 1. In other words, if the ratio is set to 0.5, the motor 
will turn at 50\% of its maximum angular velocity while moving continuously
or moving to a new goal position.\\

\noindent
{\bf Example}
\begin{lstlisting}
#include <nxt.h>

ChNXT nxt;
double ratio;

nxt.connect();

nxt.getMotorSpeedRatio(NXT_MOTORA, ratio);

printf("The speed ratio of motorA is: %lf\n", ratio);
\end{lstlisting}

\noindent
{\bf See Also}\\
\texttt{setMotorSpeedRatio(), getMotorSpeedRatio()}

%\CPlot::\DataThreeD(), \CPlot::\DataFile(), \CPlot::\Plotting(), \plotxy().\\

\noindent
\vspace{5pt}
\rule{4.5in}{0.015in}\\
\noindent
{\LARGE \texttt{ChNXT::getMotorSpeedRatios()}\index{ChNXT::getMotorSpeedRatios()}}\\
%\phantomsection
\addcontentsline{toc}{subsection}{getMotorSpeedRatios()}

\noindent
{\bf Synopsis}
\begin{lstlisting}
#include <nxt.h>
int ChNXT::getMotorSpeedRatios(double &ratio1, double &ratio2, double &ratio3);
\end{lstlisting}

\noindent
{\bf Purpose}\\
Get the speed ratio settings of all motors on the Lego Mindstorms NXT.\\

\noindent
{\bf Return Value}\\
The function returns 0 on success and non-zero otherwise.\\

\noindent
{\bf Parameters}
\vspace{-0.1in}
\begin{description}
\item               
\begin{tabular}{p{10 mm}p{145 mm}}
\texttt{ratio1} & The address of a variable to store the speed ratio of motor A.\\
\texttt{ratio2} & The address of a variable to store the speed ratio of motor B.\\
\texttt{ratio3} & The address of a variable to store the speed ratio of motor C.\\
\end{tabular}
\end{description}

\noindent
{\bf Description}\\
This function is used to retrieve all four motor speed ratio settings of a Lego
Mindstorms NXT simultaneously. The speed ratios are as a value from 0 to 1. \\

\noindent
{\bf Example}
\begin{lstlisting}
#include <nxt.h>

ChNXT nxt;
double ratio1, ratio2, ratio3;

nxt.connect();

nxt.getMotorSpeedRatio(ratio1, ratio2, ratio3);

printf("The speed ratio of motorA is: %lf\n", ratio1);
printf("The speed ratio of motorB is: %lf\n", ratio2);
printf("The speed ratio of motorC is: %lf\n", ratio3);
\end{lstlisting}

\noindent
{\bf See Also}\\
\texttt{getMotorSpeedRatio(), setMotorSpeedRatio()}

%\CPlot::\DataThreeD(), \CPlot::\DataFile(), \CPlot::\Plotting(), \plotxy().\\

\noindent
\vspace{5pt}
\rule{4.5in}{0.015in}\\
\noindent
{\LARGE \texttt{ChNXT::getMotorState()}\index{ChNXT::getMotorState()}}\\
%\phantomsection
\addcontentsline{toc}{subsection}{getMotorState()}

\noindent
{\bf Synopsis}
\begin{lstlisting}
#include <nxt.h>
int ChNXT::getMotorState(nxtMotorId_t id, nxtMotorState_t &state);
\end{lstlisting}

\noindent
{\bf Purpose}\\
Determine whether a motor is moving or not.\\

\noindent
{\bf Return Value}\\
The function returns 0 on success and non-zero otherwise.\\

\noindent
{\bf Parameters}
\vspace{-0.1in}
\begin{description}
\item               
\begin{tabular}{p{10 mm}p{145 mm}}
\texttt{id} & The motor number. This is an enumerated type 
discussed in Section \ref{sec:nxtMotorId_t} on Page~\pageref{sec:nxtMotorId_t}.\\
\texttt{state} & An integer variable which will be overwritten with the current 
state of the motor. This is an enumerated type discussed in 
Section~\ref{sec:nxtMotorState_t} on Page~\pageref{sec:nxtMotorState_t}.\\
\end{tabular}
\end{description}

\noindent
{\bf Description}\\
This function is used to determine the current state of a motor. Valid states are listed below.\\
\\
\input{motorStateTable}

\noindent
{\bf Example}
\begin{lstlisting}
#include <nxt.h>

ChNXT nxt;
nxtMotorState_t status;

nxt.connect();

nxt.getMotorState(NXT_MOTORA, status);
if(status == 0)
    printf("MotorA is not moving.\n");
else
    printf("MotorA is moving.\n");
\end{lstlisting}
%\noindent

\noindent
{\bf See Also}\\
\texttt{isMoving()}\\
%\CPlot::\DataThreeD(), \CPlot::\DataFile(), \CPlot::\Plotting(), \plotxy().\\

\noindent
\vspace{5pt}
\rule{4.5in}{0.015in}\\
\noindent
{\LARGE \texttt{ChNXT::getSensor()}\index{ChNXT::getSensor()}}\\
%\phantomsection
\addcontentsline{toc}{subsection}{getSensor()}

\noindent
{\bf Synopsis}
\vspace{-8pt}
\begin{verbatim}
#include <nxt.h>
int ChNXT::getSensor(nxtSensorId_id, double &value);
\end{verbatim}

\noindent
{\bf Purpose}\\
Retrieve a NXT sensor's current value.\\

\noindent
{\bf Return Value}\\
The function returns 0 on success and non-zero otherwise.\\

\noindent
{\bf Parameters}\\
\vspace{-0.1in}
\begin{description}
\item               
\begin{tabular}{p{15 mm}p{145 mm}}
\texttt{id} & The sensor port. This is an enumerated type 
discussed in Section \ref{sec:nxtSensorId_t} on page
\pageref{sec:nxtSensorId_t}.\\
\texttt{value} & A variable to store the current value of the NXT 
sensor. The contents of this variable will be overwritten with the 
sensor's value.  \\
\end{tabular}
\end{description}

\noindent
{\bf Description}\\
This function gets the current sensor's value of a NXT.\\

\noindent
{\bf Example}\\
\begin{verbatim}
#include <nxt.h>
#include <stdio.h>

ChNXT nxt;
int value;

nxt.connect();

nxt.setSensor(NXT_SENSORPORT4, NXT_SENSORTYPE_ULTRASONIC,
		NXT_SENSORMODE_RAWMODE);

nxt.getSensor(NXT_SENSORPORT4, value);

printf("The value of sensor 4 is: %d\n", value);

nxt.disconnect();
\end{verbatim}

\noindent
{\bf See Also}\\

%\CPlot::\DataThreeD(), \CPlot::\DataFile(), \CPlot::\Plotting(), \plotxy().\\

\noindent
\vspace{5pt}
\rule{4.5in}{0.015in}\\
\noindent
{\LARGE \texttt{ChNXT::humanoidWalkBackward()}\index{ChNXT::humanoidWalkBackward()}}\\
{\LARGE \texttt{ChNXT::humanoidWalkBackwardNB()}\index{ChNXT::humanoidWalkBackwardNB()}}\\
%\phantomsection
\addcontentsline{toc}{subsection}{humanoidWalkBackward()}
\addcontentsline{toc}{subsection}{humanoidWalkBackwardNB()}

\noindent
{\bf Synopsis}
\begin{lstlisting}
#include <mobot.h>
int ChNXT::humanoidWalkBackward(double angle);
int ChNXT::humanoidWalkBackwardNB(double angle);
\end{lstlisting}

\noindent
{\bf Purpose}\\
Make the Lego Mindstorms NXT in the humanoid configuration walk backward.\\

\noindent
{\bf Return Value}\\
The function returns 0 on success and non-zero otherwise.\\

\noindent
{\bf Parameters}\\
\vspace{-0.1in}
\begin{description}
\item               
\begin{tabular}{p{15 mm}p{145 mm}}
\texttt{angle} & The angle to turn the motors, specified in degrees.\\
\end{tabular}
\end{description}

\noindent
{\bf Description}\\
\vspace{-12pt}
\begin{quote}
{\bf ChNXT::humanoidWalkBackward()}\\
This function is used to make the Lego Mindstorms walk backward in the humanoid
configuration. The amount to roll the motors is specified by the argument,
\texttt{angle}.

{\bf ChNXT::humanoidWalkBackwardNB()}\\
This function is used to make the Lego Mindstorms walk backward in the humanoid
configuration. The amount to roll the motors is specified by the argument,
\texttt{angle}.

This function has both a blocking and non-blocking version.
The blocking version, \texttt{humanoidWalkBackward()}, will block until the
robot motion has completed. The non-blocking version, \texttt{humanoidWalkBackwardNB()},
will return immediately, and the motion will be performed asynchronously.\\
\end{quote}

\noindent
{\bf Example}
\begin{lstlisting}
#include <nxt.h>

ChNXT nxt;

nxt.connect();

/* Blocking function */
nxt.humanoidWalkBackward(360);

/* Non-blocking function */
nxt.humanoidWalkBackwardNB(360);
nxt.humanoidMotionWait();
\end{lstlisting}

\noindent
{\bf See Also}\\
\texttt{humanoidWalkBackward()}

%\CPlot::\DataThreeD(), \CPlot::\DataFile(), \CPlot::\Plotting(), \plotxy().\\

\noindent
\vspace{5pt}
\rule{4.5in}{0.015in}\\
\noindent
{\LARGE \texttt{ChNXT::humanoidWalkForward()}\index{ChNXT::humanoidWalkForward()}}\\
{\LARGE \texttt{ChNXT::humanoidWalkForwardNB()}\index{ChNXT::humanoidWalkForwardNB()}}\\
%\phantomsection
\addcontentsline{toc}{subsection}{humanoidWalkForward()}
\addcontentsline{toc}{subsection}{humanoidWalkForwardNB()}

\noindent
{\bf Synopsis}
\vspace{-8pt}
\begin{verbatim}
#include <mobot.h>
int ChNXT::humanoidWalkForward(double angle);
int ChNXT::humanoidWalkForwardNB(double angle);
\end{verbatim}

\noindent
{\bf Purpose}\\
Make the Lego Mindstorms NXT in the humanoid configuration walk forward.\\

\noindent
{\bf Return Value}\\
The function returns 0 on success and non-zero otherwise.\\

\noindent
{\bf Parameters}\\
\vspace{-0.1in}
\begin{description}
\item               
\begin{tabular}{p{15 mm}p{145 mm}}
\texttt{angle} & The angle to turn the joints, specified in degrees.\\
\end{tabular}
\end{description}

\noindent
{\bf Description}\\
\vspace{-12pt}
\begin{quote}
{\bf ChNXT::humanoidWalkForward()}\\
This function is used to make the Lego Mindstorms walk forward in the humanoid
configuration. The amount to roll the joints is specified by the argument,
\texttt{angle}.

{\bf ChNXT::humanoidWalkForwardNB()}\\
This function is used to make the Lego Mindstorms walk forward in the humanoid
configuration. The amount to roll the joints is specified by the argument,
\texttt{angle}.

This function has both a blocking and non-blocking version.
The blocking version, \texttt{humanoidWalkForward()}, will block until the
robot motion has completed. The non-blocking version, \texttt{humanoidWalkForwardNB()},
will return immediately, and the motion will be performed asynchronously.\\
\end{quote}

\noindent
{\bf Example}
\begin{verbatim}
#include <nxt.h>

ChNXT nxt;

nxt.connect();

/* Blocking function */
nxt.humanoidWalkForward(360);

/* Non-blocking function */
nxt.humanoidWalkForwardNB(360);
nxt.humanoidMotionWait();

nxt.disconnect();
\end{verbatim}

\noindent
{\bf See Also}\\
\texttt{humanoidWalkBackward()}

%\CPlot::\DataThreeD(), \CPlot::\DataFile(), \CPlot::\Plotting(), \plotxy().\\

\noindent
\vspace{5pt}
\rule{4.5in}{0.015in}\\
\noindent
{\LARGE \texttt{ChNXT::humanoidMotionWait()}\index{ChNXT::humanoidMotionWait()}}\\
%\phantomsection
\addcontentsline{toc}{subsection}{humanoidMotionWait()}

\noindent
{\bf Synopsis}
\vspace{-8pt}
\begin{verbatim}
#include <nxt.h>
int ChNXT::humanoidMotionWait();
\end{verbatim}

\noindent
{\bf Purpose}\\
Wait for a motion to complete execution in humanoid configuration.\\

\noindent
{\bf Return Value}\\
The function returns 0 on success and non-zero otherwise.\\

\noindent
{\bf Description}\\
This function is used to wait for a motion function to fully complete its cycle.\\


\noindent
{\bf Example}\\
\begin{verbatim}
#include <nxt.h>

ChNXT nxt;

nxt.connect();

/* Non-blocking function */
nxt.humanoidRotateRightNB(360);

/* wait until non-blocking motion stops */
nxt.humanoidMotionWait();

nxt.disconnect();
\end{verbatim}

\noindent
{\bf See Also}\\
\texttt{humanoidWalkForward()}, \texttt{humanoidWalkBackward()}



%\CPlot::\DataThreeD(), \CPlot::\DataFile(), \CPlot::\Plotting(), \plotxy().\\

\noindent
\vspace{5pt}
\rule{4.5in}{0.015in}\\
\noindent
{\LARGE \texttt{ChNXT::isConnected()} \index{ChNXT::isConnected()}}\\

\addcontentsline{toc}{subsection}{isConnect()}

\noindent
{\bf Synopsis}
\vspace{-8pt}
\begin{verbatim}
#include <nxt.h>
int ChNXT::isConnected(void);
\end{verbatim}

\noindent
{\bf Purpose}\\
Check to see if currently connected to a Lego Mindstorms NXT via Bluetooth.

\noindent
{\bf Return Value}\\
The function returns zero if it is not currently connected to a Lego Mindstorms NXT or if an error has occured, or 1 if the Lego Mindstorms NXT is connected.\\

\noindent
{\bf Parameters}\\
None.\\

\noindent
{\bf Description}\\
This function is used to check if the software is currently connected to a Lego Mindstorms NXT.\\

\noindent
{\bf Example}
\begin{verbatim}
#include <nxt.h>

ChNXT nxt;
int connectStatus;

nxt.connect();
connectStatus = nxt.isConnected();

if(connectStatus)
    printf("Connected!\n");
else
    printf("Not connected!\n");
\end{verbatim}

\noindent
{\bf See Also}\\
%%%%% END OF ChNXT::isConnect %%%%%

\noindent
\vspace{5pt}
\rule{4.5in}{0.015in}\\
\noindent
{\LARGE \texttt{ChNXT::isMoving()} \index{ChNXT::isMoving()}}\\

\addcontentsline{toc}{subsection}{isMoving()}

\noindent
{\bf Synopsis}
\begin{lstlisting}
#include <nxt.h>
int ChNXT::isMoving(void);
\end{lstlisting}

\noindent
{\bf Purpose}\\
Check to see if a Lego Mindstorms NXT is currently moving any of its motors.\\

\noindent
{\bf Return Value}\\
This function returns 0 if none of the motors are being driven or 
if an error has occured, or 1 if any motor is being driven.\\

\noindent
{\bf Parameters}\\
None.\\

\noindent
{\bf Description}\\
This function is used to determine if a robot is currently moving any of its motors.\\

\noindent
{\bf Example}
\begin{lstlisting}
#include <nxt.h>

ChNXT nxt;
int moveStatus;

nxt.connect();
moveStatus = isMoving(NXT_MOTORA);

if(moveStatus)
    printf("MotorA is moving!\n");
else
    printf("MotorA is not moving!\n");
\end{lstlisting}

\noindent
{\bf See Also}\\
\texttt{getMotorState()}\\
%%%%% END OF ChNXT::connect %%%%%

\noindent
\vspace{5pt}
\rule{4.5in}{0.015in}\\
\noindent
{\LARGE \texttt{ChNXT::move()} \index{ChNXT::move()}}\\
{\LARGE \texttt{ChNXT::moveNB()} \index{ChNXT::moveNB()}}\\

\addcontentsline{toc}{subsection}{move()}
\addcontentsline{toc}{subsection}{moveNB()}

\noindent
{\bf Synopsis}
\vspace{-8pt}
\begin{verbatim}
#include <nxt.h>
int ChNXT::move(double degrees1, double degrees2, double degrees3);
int ChNXT::moveNB(double degrees1, double degrees2, double degrees3);
\end{verbatim}

\noindent
{\bf Purpose}\\
Move all joints of Lego Mindstorms NXT by specified degrees.\\

\noindent
{\bf Return Value}\\
The function returns 0 on success and non-zero otherwise.\\

\noindent
{\bf Parameters}\\
\vspace{-0.1in}
\begin{description}
\item               
\begin{tabular}{p{15 mm}p{125 mm}}
\texttt{angle1} & The amount to move joint 1 relative to the current position. \\
\texttt{angle2} & The amount to move joint 2 relative to the current position. \\
\texttt{angle3} & The amount to move joint 3 relative to the current position. \\
\end{tabular}
\end{description}

\noindent
{\bf Description}\\
\vspace{-12pt}
\begin{quote}
{\bf ChNXT::move()}\\
This function moves all of the joints of a Lego Mindstorms NXT by 
the specified number of degrees from their current positions. 

\noindent
{\bf ChNXT::moveNB()}\\
This function moves all of the joints of a Lego Mindstorms NXT by 
the specified number of degrees from their current positions. 

The function \texttt{moveNB()} is the non-blocking version of the 
\texttt{move()} function, which means that the function will 
return immediately and the physical robot motion will occur 
asynchronously. For more information on blocking and non-blocking 
functions, please refer to Section \ref{sec:block_nonblock} on page 
\pageref{sec:block_nonblock}.\\
\end{quote}

\noindent
{\bf Example}
\begin{verbatim}
#include <nxt.h> 

ChNXT nxt;

if (nxt.connectWithAddress("00:16:53:12:e7:80")){
    printf("Connection to NXT has failed.");
    exit(0);
}
 
/* setup speed for all three joints */
nxt.setJointSpeeds(60, 40, 40);

/* move by the non-blocking function*/
nxt.moveNB(360, 360, 360);
nxt.moveWait();

/* move by the blocking function*/
nxt.move(360, 360, 360);
\end{verbatim}

\noindent
{\bf See Also}\\
{\tt moveTo(), moveToNB()}\\
%%%%% END OF ChNXT::move %%%%%

\noindent
\vspace{5pt}
\rule{4.5in}{0.015in}\\
\noindent
{\LARGE \texttt{ChNXT::moveContinuousNB()} \index{ChNXT::moveContinuousNB()}}\\

\addcontentsline{toc}{subsection}{moveContinuousNB()}

\noindent
{\bf Synopsis}
\begin{lstlisting}
#include <nxt.h>
int ChNXT::moveContinuousNB(nxtJointState_t dirA, 
                            nxtJointState_t dirB, 
                            nxtJointState_t dirC); 
\end{lstlisting}

\noindent
{\bf Purpose}\\
Move the joints of a robot continuously in the specified directions.\\

\noindent
{\bf Return Value}\\
The function returns 0 on success and non-zero otherwise.\\

\noindent
{\bf Parameters}\\
Each parameter specifies the direction the joint should move. The types
are enumerated in \texttt{mobot.h} and have the following values:\\
\\
\noindent
\begin{tabular}{p{1.75in}p{4.5in}} \hline 
Value & Description \\
\hline 
\texttt{NXT\_FORWARD} & This value indicates that the joint is currently moving forward. \\
\texttt{NXT\_BACKWARD}& This value indicates that the joint is currently moving backward.\\
\hline
\end{tabular}\\


\noindent
More documentation about these types may be found at Section
\ref{sec:nxtJointState_t} on page
\pageref{sec:nxtJointState_t}.

\noindent
{\bf Description}\\
This function causes joints of a robot to begin moving at the 
previously set speed. The joints will continue moving until the 
joint hits a joint limit, or the joint is stopped by setting the 
speed to zero. This function is a non-blocking function.\\

\noindent
{\bf Example}
\begin{lstlisting}
#include <nxt.h> 

ChNXT nxt;

if (nxt.connectWithAddress("00:16:53:12:e7:80")){
    printf("Connection to NXT has failed.");
    exit(0);
}
 
/* setup speed for all three joints */
nxt.setJointSpeeds(60, 40, 40);

/* move all joints on the Mindstorm NXT */
nxt.moveContinuousNB(NXT_FORWARD, NXT_FORWARD, NXT_FORWARD);
\end{lstlisting}

%\noindent
%{\bf Output}\\
%None.\\
%\\
\noindent
{\bf See Also}\\
%%%%% END OF ChNXT::moveContinuousNB %%%%%

\noindent
\vspace{5pt}
\rule{4.5in}{0.015in}\\
\noindent
{\LARGE \texttt{ChNXT::moveContinuousTime()} \index{ChNXT::moveContinuousTime()}}\\

\addcontentsline{toc}{subsection}{moveContinuousTime()}

\noindent
{\bf Synopsis}
\begin{lstlisting}
#include <nxt.h>
int ChNXT::moveContinuousTime(nxtMotorState_t dirA, 
                              nxtMotorState_t dirB, 
                              nxtMotorState_t dirC, 
                              double seconds);
\end{lstlisting}

\noindent
{\bf Purpose}\\
Move the motors of a robot continuously in the specified directions for some amount of time.\\

\noindent
{\bf Return Value}\\
The function returns 0 on success and non-zero otherwise.\\

\noindent
{\bf Parameters}\\
Each direction parameter specifies the direction the motor should 
move. The types are enumerated in \texttt{nxt.h} and have the 
following values:\\ \input{motorStateTable} 
\noindent
The \texttt{seconds} parameter is the time to perform the movement, in seconds.
\\

\noindent
{\bf Description}\\
This function causes motors of a robot to begin moving. The motors
will continue moving until the motor hits a motor limit, or the 
time specified in the \texttt{seconds} parameter is reached. This 
function will block until the motion is completed.\\

\noindent
{\bf Example}
\begin{lstlisting}
#include <nxt.h> 

ChNXT nxt;

if (nxt.connectWithAddress("00:16:53:12:e7:80")){
    printf("Connection to NXT has failed.");
    exit(0);
}
 
/* setup speed for all three motors */
nxt.setMotorSpeedRatios(0.4, 0.4, 0.4);

/* move all motors on the Mindstorm NXT in 10 seconds*/
nxt.moveContinuousTime(NXT_FORWARD, NXT_FORWARD, 
                       NXT_FORWARD, 10);
\end{lstlisting}

%\noindent
%{\bf Output}\\
%None.\\
%\\
\noindent
{\bf See Also}\\
%%%%% END OF ChNXT::moveContinuousTime %%%%%

\noindent
\vspace{5pt}
\rule{4.5in}{0.015in}\\
\noindent
{\LARGE \texttt{ChNXT::moveMotor()} \index{ChNXT::moveMotor()}}\\
{\LARGE \texttt{ChNXT::moveMotorNB()} \index{ChNXT::moveMotorNB()}}\\

\addcontentsline{toc}{subsection}{moveMotor()}
\addcontentsline{toc}{subsection}{moveMotorNB()}

\noindent
{\bf Synopsis}
\begin{lstlisting}
#include <nxt.h>
int ChNXT::moveMotor(nxtMotorId_t id, double angle);
int ChNXT::moveMotorNB(nxtMotorId_t id, double angle);
\end{lstlisting}

\noindent
{\bf Purpose}\\
Move a motor on the robot by a specified angle with respect to the
current position.\\

\noindent
{\bf Return Value}\\
The function returns 0 on success and non-zero otherwise.\\

\noindent
{\bf Parameters}\\
\vspace{-0.1in}
\begin{description}
\item               
\begin{tabular}{p{20 mm}p{135 mm}}
\texttt{id} & The motor number to move. \\
\texttt{angle} & The desired angle the motor need to move. \\
\end{tabular}
\end{description}

\noindent
{\bf Description}\\
\vspace{-12pt}
\begin{quote}
{\bf ChNXT::moveMotor()}\\
This function commands the motor to move by an angle relative to 
the motor's current position at the motors current speed setting.
The current motor speed ratio may be set with the 
\texttt{setMotorSpeedRatio()} member function. Please note that if
the motor speed is set to zero, the motor will not move after 
calling the \texttt{moveMotor()} function. 

{\bf ChNXT::moveMotorNB()}\\
This function commands the motor to move by an angle relative to 
the motor's current position at the motors current speed setting.
The current motor speed ratio may be set with the
\texttt{setMotorSpeedRatio()} member function. Please note that if
the motor speed is set to zero, the motor will not move after 
calling the \texttt{moveMotorNB()} function. 

The function \texttt{moveMotorNB()} is the non-blocking version of
the \texttt{moveMotor()} function, which means that the function 
will return immediately and the physical robot motion will occur 
asynchronously. For more details on blocking and non-blocking 
functions, please refer to Section \ref{sec:block_nonblock} on page 
\pageref{sec:block_nonblock}.\\
\end{quote}

\noindent
{\bf Example}
\begin{lstlisting}
#include <nxt.h> 

ChNXT nxt;

if (nxt.connectWithAddress("00:16:53:12:e7:80")){
    printf("Connection to NXT has failed.");
    exit(0);
}
 
/* setup speed for motor1 */
nxt.setMotorSpeedRatio(NXT_MOTORA, 40);

/* move motor1 360 degrees */
nxt.moveMotorNB(NXT_MOTORA, 360);
nxt.moveMotorWait();

/* move motor1 360 degrees */
nxt.moveMotor(NXT_MOTORA, 360);
\end{lstlisting}

%\noindent
%{\bf Output}\\
%None.\\
%\\
\noindent
{\bf See Also}\\
%%%%% END OF ChNXT::moveMotor %%%%%

\noindent
\vspace{5pt}
\rule{4.5in}{0.015in}\\
\noindent
{\LARGE \texttt{ChNXT::moveMotorContinuousNB()} \index{ChNXT::moveMotorContinuousNB()}}\\

\addcontentsline{toc}{subsection}{moveMotorContinuousNB()}

\noindent
{\bf Synopsis}
\begin{lstlisting}
#include <nxt.h>
int ChNXT::moveMotorContinuousNB(robotMotorId_t id, int dir);
\end{lstlisting}

\noindent
{\bf Purpose}\\
Move the specific motor on the Mindstorm NXT continuously in a direction.\\

\noindent
{\bf Return Value}\\
The function returns 0 on success and non-zero otherwise.\\

\noindent
{\bf Parameters}\\
\vspace{-0.1in}
\begin{description}
\item               
\begin{tabular}{p{15 mm}p{125 mm}}
\texttt{id}      &The motor of the Mindstorm NXT.\\
\texttt{dir}     &The move direction of the motor. 1 means positive and 0 means negative direction.\\
\end{tabular}
\end{description}

\noindent
{\bf Description}\\
This function is used to rotate the specific motor of the Mindstorm NXT. The motor will move continuously until the function 
\texttt{stopOneMotor()} is called. The variable \texttt{dir} indicates the move direction of the motor. \\


\noindent
{\bf Example}
\begin{lstlisting}
#include <nxt.h> 

ChNXT nxt;

if (nxt.connectWithAddress("00:16:53:12:e7:80")){
    printf("Connection to NXT has failed.");
    exit(-1);
}
 
/* setup speed for all three motors */
nxt.setMotorSpeeds(60, 40, 40);

/* move the motor 1 on the Mindstorm NXT */
nxt.moveMotorContinuousNB(NXT_MOTORA, NXT_FORWARD);
\end{lstlisting}

%\noindent
%{\bf Output}\\
%None.\\
%\\
\noindent
{\bf See Also}\\
%%%%% END OF ChNXT::moveMotorContinuousNB %%%%%

\noindent
\vspace{5pt}
\rule{4.5in}{0.015in}\\
\noindent
{\LARGE \texttt{ChNXT::moveMotorTo()} \index{ChNXT::moveMotorTo()}}\\
{\LARGE \texttt{ChNXT::moveMotorToNB()} \index{ChNXT::moveMotorToNB()}}\\

\addcontentsline{toc}{subsection}{moveMotorTo()}
\addcontentsline{toc}{subsection}{moveMotorToNB()}

\noindent
{\bf Synopsis}
\begin{lstlisting}
#include <nxt.h>
int ChNXT::moveMotorTo(nxtMotorId_t id, double angle);
int ChNXT::moveMotorToNB(nxtMotorId_t id, double angle);
\end{lstlisting}

\noindent
{\bf Purpose}\\
Move a motor on the Lego Mindstorms NXT to an absolute position.\\

\noindent
{\bf Return Value}\\
The function returns 0 on success and non-zero otherwise.\\

\noindent
{\bf Parameters}\\
\vspace{-0.1in}
\begin{description}
\item               
\begin{tabular}{p{10 mm}p{145 mm}}
\texttt{id} & The motor number to move. \\
\texttt{angle}&The absolute angle in degrees to move the motor to.\\
\end{tabular}
\end{description}

\noindent
{\bf Description}\\
\vspace{-12pt}
\begin{quote}
{\bf ChNXT::moveMotorTo()}\\
This function commands the motor to move to a position specified 
in degrees at the current motor's speed. The current motor speed 
may be set with the \texttt{setMotorSpeedRatio()} member function.
Please note that if the motor speed is set to zero, the motor will
not move after calling the \texttt{moveMotorTo()} function. 

{\bf ChNXT::moveMotorToNB()}\\
This function commands the motor to move to a position specified 
in degrees at the current motor's speed. The current motor speed 
may be set with the \texttt{setMotorSpeedRatio()} member function.
Please note that if the motor speed is set to zero, the motor will
not move after calling the \texttt{moveMotorToNB()} function. 

The function \texttt{moveMotorToNB()} is the non-blocking version 
of the \texttt{moveMotorTo()} function, which means that the 
function will return immediately and the physical robot motion 
will occur asynchronously. For more details on blocking and 
non-blocking functions, please refer to Section \ref{sec:block_nonblock}
on page \pageref{sec:block_nonblock}.\\
\end{quote}

\noindent
{\bf Example}
\begin{lstlisting}
#include <nxt.h> 

ChNXT nxt;

if (nxt.connectWithAddress("00:16:53:12:e7:80")){
    printf("Connection to NXT has failed.");
    exit(-1);
}
 
/* setup speed for all three motors */
nxt.setMotorSpeedRatio(NXT_MOTORA, 40);

/* move motor1 360 degrees with the non-blocking function */
nxt.moveMotorToNB(NXT_MOTORA, 360);
nxt.moveMotorWait(NXT_MOTORA);

/* move motor1 360 degrees with the blocking function */
nxt.moveMotorTo(NXT_MOTORA, 360);
\end{lstlisting}

%\noindent
%{\bf Output}\\
%None.\\
%\\
\noindent
{\bf See Also}\\
%%%%% END OF ChNXT::moveMotorTo %%%%%

\noindent
\vspace{5pt}
\rule{4.5in}{0.015in}\\
\noindent
{\LARGE \texttt{ChNXT::moveMotorWait()}\index{ChNXT::moveMotorWait()}}\\
%\phantomsection
\addcontentsline{toc}{subsection}{moveMotorWait()}

\noindent
{\bf Synopsis}
\begin{lstlisting}
#include <nxt.h>
int ChNXT::moveMotorWait(nxtMotorId_t id);
\end{lstlisting}

\noindent
{\bf Purpose}\\
Wait for a motor to stop moving.\\

\noindent
{\bf Return Value}\\
The function returns 0 on success and non-zero otherwise.\\

\noindent
{\bf Parameters}
\vspace{-0.1in}
\begin{description}
\item               
\begin{tabular}{p{10 mm}p{145 mm}}
\texttt{id} & The motor number to wait for. \\
\end{tabular}
\end{description}

\noindent
{\bf Description}\\
This function is used to wait for a motor motion to finish. Functions such as
\texttt{moveNB()} and \texttt{moveMotorNB()} do not wait for a motor to finish
moving before continuing to allow multiple motors to move at the same time. The
\texttt{moveMotorWait()} function is used to wait for a 
robotic motor motion to complete.

Please note that if this function is called after a motor has been commanded to
turn indefinitely, this function may never return and your program may hang.\\

\noindent
{\bf Example}
%Please see the example in Section \ref{sec:democode} on page \pageref{sec:democode}.\\
%\noindent

\noindent
{\bf See Also}\\
\texttt{moveWait()}

%\CPlot::\DataThreeD(), \CPlot::\DataFile(), \CPlot::\Plotting(), \plotxy().\\

\noindent
\vspace{5pt}
\rule{4.5in}{0.015in}\\
\noindent
{\LARGE \texttt{ChNXT::moveTo()} \index{ChNXT::moveTo()}}\\
{\LARGE \texttt{ChNXT::moveToNB()} \index{ChNXT::moveToNB()}}\\

\addcontentsline{toc}{subsection}{moveTo()}
\addcontentsline{toc}{subsection}{moveToNB()}

\noindent
{\bf Synopsis}
\vspace{-8pt}
\begin{verbatim}
#include <nxt.h>
int ChNXT::moveTo(double angle1, double angle2, double angle3, double angle4);
int ChNXT::moveToNB(double angle1, double angle2, double angle3, double angle4);
\end{verbatim}

\noindent
{\bf Purpose}\\
Move all joints of Lego Mindstorms NXT to the specified position.\\

\noindent
{\bf Return Value}\\
The function returns 0 on success and non-zero otherwise.\\

\noindent
{\bf Parameters}\\
\vspace{-0.1in}
\begin{description}
\item               
\begin{tabular}{p{15 mm}p{105 mm}}
\texttt{angle1} & The absolute position to move joint 1, expressed in degrees. \\
\texttt{angle2} & The absolute position to move joint 2, expressed in degrees. \\
\texttt{angle3} & The absolute position to move joint 3, expressed in degrees. \\
\end{tabular}
\end{description}

\noindent
{\bf Description}\\
\vspace{-12pt}
\begin{quote}
{\bf ChNXT::moveTo()}\\
This function moves all of the joints of a robot to the specified 
absolute positions. 

{\bf ChNXT::moveToNB()}\\
This function moves all of the joints of a robot to the specified 
absolute positions. 

The function \texttt{moveToNB()} is the non-blocking version of
the \texttt{moveTo()} function, which means that the function will
return immediately and the physical robot motion will occur 
asynchronously. For more details on blocking and non-blocking 
functions, please refer to Section \ref{sec:block_nonblock} on page 
\pageref{sec:block_nonblock}.\\
\end{quote}

\noindent
{\bf Example}
\begin{verbatim}
#include <nxt.h> 
#include <stdio.h>

ChNXT nxt;

if (nxt.connectWithAddress("00:16:53:12:e7:80")){
    printf("Connection to NXT has failed.");
    exit(0);
}
 
/* setup speed for all three joints */
nxt.setJointSpeeds(40, 40, 40);

/* move by the non-blocking function*/
nxt.moveToNB(360, 360, 360);
nxt.moveWait();

/* move by the blocking function*/
nxt.moveTo(360, 360, 360);

nxt.disconnect();
\end{verbatim}

\noindent
{\bf See Also}\\
%%%%% END OF ChNXT::move %%%%%

\noindent
\vspace{5pt}
\rule{4.5in}{0.015in}\\
\noindent
{\LARGE \texttt{ChNXT::moveToZero()} \index{ChNXT::moveToZero()}}\\
{\LARGE \texttt{ChNXT::moveToZeroNB()} \index{ChNXT::moveToZeroNB()}}\\

\addcontentsline{toc}{subsection}{moveToZero()}
\addcontentsline{toc}{subsection}{moveToZeroNB()}

\noindent
{\bf Synopsis}
\begin{lstlisting}
#include <nxt.h>
int ChNXT::moveToZero();
int ChNXT::moveToZeroNB();
\end{lstlisting}

\noindent
{\bf Purpose}\\
Move all joints of Lego Mindstorms NXT to their absolute zero position. \\

\noindent
{\bf Return Value}\\
The function returns 0 on success and non-zero otherwise.\\

\noindent
{\bf Parameters}\\
None.\\

\noindent
{\bf Description}\\
\vspace{-12pt}
\begin{quote}
{\bf ChNXT::moveToZero()}\\
This function moves all of the joints of a NXT to their zero 
position.

{\bf ChNXT::moveToZeroNB()}\\
This function moves all of the joints of a NXT to their zero 
position.

The function \texttt{moveToZeroNB()} is the non-blocking version 
of the \texttt{moveToZero()} function, which means that the 
function will return immediately and the physical robot motion 
will occur asynchronously. For more details on blocking and 
non-blocking functions, please refer to Section \ref{sec:block_nonblock}
on page \pageref{sec:block_nonblock}.\\
\end{quote}

\noindent
{\bf Example}
\begin{lstlisting}
#include <nxt.h> 

ChNXT nxt;

if (nxt.connectWithAddress("00:16:53:12:e7:80")){
    printf("Connection to NXT has failed.");
    exit(-1);
}
 
/* setup speed for all three joints */
nxt.setJointSpeeds(60, 40, 40);

/* move to zero by the non-blocking function*/
nxt.moveToZeroNB();
nxt.moveWait();

/* move to zero by the blocking function*/
nxt.moveToZero();
\end{lstlisting}

%\noindent
%{\bf Output}\\
%None.\\
%\\
\noindent
{\bf See Also}\\
%%%%% END OF ChNXT::moveToZero %%%%%

\noindent
\vspace{5pt}
\rule{4.5in}{0.015in}\\
\noindent
{\LARGE \texttt{ChNXT::moveWait()}\index{ChNXT::moveWait()}}\\
%\phantomsection
\addcontentsline{toc}{subsection}{moveWait()}

\noindent
{\bf Synopsis}
\begin{lstlisting}
#include <nxt.h>
int ChNXT::moveWait();
\end{lstlisting}

\noindent
{\bf Purpose}\\
Wait for all motors to stop moving.\\

\noindent
{\bf Return Value}\\
The function returns 0 on success and non-zero otherwise.\\

\noindent
{\bf Description}\\
This function is used to wait for all motor motions to finish. Functions such as
\texttt{move()} and \texttt{moveTo()} do not wait for a motor to finish
moving before continuing to allow multiple motors to move at the same time. The
\texttt{moveWait()} function is used to wait for robotic motions to complete.\\

\noindent
Please note that if this function is called after a motor has been commanded to
turn indefinitely, this function may never return and your program may hang.\\

\noindent
{\bf Example}
\begin{lstlisting}
#include <nxt.h> 

ChNXT nxt;

if (!nxt.connectWithAddress("00:16:53:12:e7:80")){
    printf("Connection to NXT has failed.");
    exit(-1);
}
 
/* setup speed for all three motors */
nxt.setMotorSpeeds(60, 40, 40);

/* move by the non-blocking function*/
nxt.moveNB(360, 360, 360);

/* wait until motors stop moving */
nxt.moveWait();
\end{lstlisting}

%See the sample program in Section \ref{sec:democode} on page \pageref{sec:democode}.

\noindent
{\bf See Also}\\
\texttt{moveMotorWait()}

%\CPlot::\DataThreeD(), \CPlot::\DataFile(), \CPlot::\Plotting(), \plotxy().\\

\input{api/set/setMotorToZero}
%\input{api/set/setMotorZeros}
%\noindent
\vspace{5pt}
\rule{4.5in}{0.015in}\\
\noindent
{\LARGE \texttt{ChNXT::resetToZero()} \index{ChNXT::resetToZero()}}\\

\addcontentsline{toc}{subsection}{resetToZero()}

\noindent
{\bf Synopsis}
\vspace{-8pt}
\begin{verbatim}
#include <nxt.h>
int ChNXT::resetToZero(void);
\end{verbatim}

\noindent
{\bf Purpose}\\
Reset the tachometer count for all motors.\\

\noindent
{\bf Return Value}\\
The function returns 0 on success and non-zero otherwise.\\

\noindent
{\bf Parameters}\\
None.\\

\noindent
{\bf Description}\\
This function is used to reset the tachometer count for 
all joints of the Mindstorms NXT.\\

\noindent
{\bf Example}
\begin{verbatim}
#include <nxt.h> 
#include <stdio.h>

ChNXT nxt;

if (nxt.connectWithAddress("00:16:53:12:e7:80")){
    printf("Connection to NXT has failed.");
    exit(0);
}
    
nxt.resetToZero();

nxt.disconnect();
\end{verbatim}

%\noindent
%{\bf Output}\\
%None.\\
%\\
\noindent
{\bf See Also}\\
\texttt{setJointZero()}\\
%%%%% END OF ChNXT::resetToZero %%%%%

\input{api/set/setToZero}
%\noindent
\vspace{5pt}
\rule{4.5in}{0.015in}\\
\noindent
{\LARGE \texttt{ChNXT::setMotorAbsoluteZero()} \index{ChNXT::setMotorAbsoluteZero()}}\\

\addcontentsline{toc}{subsection}{setMotorAbsoluteZero()}

\noindent
{\bf Synopsis}
\begin{lstlisting}
#include <nxt.h>
int ChNXT::setMotorAbsoluteZero(nxtMotorId_t id);
\end{lstlisting}

\noindent
{\bf Purpose}\\
Reset the absolute tachometer count for a single motor on Mindstorms NXT.\\

\noindent
{\bf Return Value}\\
The function returns 0 on success and non-zero otherwise.\\

\noindent
{\bf Parameters}\\
\vspace{-0.1in}
\begin{description}
\item
\begin{tabular}{p{20mm}p{135mm}}
\texttt{id} &The port of the motor locate on the Mindstorm NXT.\\
\end{tabular}
\end{description}

\noindent
{\bf Description}\\
This function is used to reset the absolute tachometer count for a motor on the Mindstorms NXT.\\

\noindent
{\bf Example}
\begin{lstlisting}
#include <nxt.h> 

ChNXT nxt;

if (!nxt.connectWithAddress("00:16:53:12:e7:80")){
    printf("Connection to NXT has failed.");
    exit(-1);
}
    
nxt.setMotorAbsoluteZero(NXT_MOTORA);
\end{lstlisting}

\noindent
{\bf Output}\\
None.\\
\\
%%%%% END OF ChNXT::setMotorRelativeZero %%%%%

%\input{api/set/setMotorAbsoluteZeros}
%\noindent
\vspace{5pt}
\rule{4.5in}{0.015in}\\
\noindent
{\LARGE \texttt{ChNXT::setMotorRelativeZero()} \index{ChNXT::setMotorRelativeZero()}}\\

\addcontentsline{toc}{subsection}{setMotorRelativeZero()}

\noindent
{\bf Synopsis}
\begin{lstlisting}
#include <nxt.h>
int ChNXT::setMotorRelativeZero(nxtMotorId_t id);
\end{lstlisting}

\noindent
{\bf Purpose}\\
Reset the relative tachometer count for a single motor on Mindstorms NXT.\\

\noindent
{\bf Return Value}\\
The function returns 0 on success and non-zero otherwise.\\

\noindent
{\bf Parameters}\\
\vspace{-0.1in}
\begin{description}
\item
\begin{tabular}{p{20mm}p{135mm}}
\texttt{id} &The port of a motor located on the Mindstorm NXT.\\
\end{tabular}
\end{description}

\noindent
{\bf Description}\\
This function is used to reset the relative tachometer count for a motor on the Mindstorms NXT.\\

\noindent
{\bf Example}
\begin{lstlisting}
#include <nxt.h> 

ChNXT nxt;

if (!nxt.connectWithAddress("00:16:53:12:e7:80")){
    printf("Connection to NXT has failed.");
    exit(-1);
}
    
nxt.setMotorRelativeZero(NXT_MOTORA);
\end{lstlisting}

\noindent
{\bf Output}\\
None.\\
\\
%%%%% END OF ChNXT::setMotorRelativeZero %%%%%

%\noindent
\vspace{5pt}
\rule{4.5in}{0.015in}\\
\noindent
{\LARGE \texttt{ChNXT::setMotorRelativeZeros()} \index{ChNXT::setMotorRelativeZeros()}}\\

\addcontentsline{toc}{subsection}{setMotorRelativeZeros()}

\noindent
{\bf Synopsis}
\begin{lstlisting}
#include <nxt.h>
int ChNXT::setMotorRelativeZero(void);
\end{lstlisting}

\noindent
{\bf Purpose}\\
Reset the relative tachometer count for all motors.\\

\noindent
{\bf Return Value}\\
The function returns 0 on success and non-zero otherwise.\\

\noindent
{\bf Parameters}\\
None.\\

\noindent
{\bf Description}\\
This function is used to reset the relative tachometer count for 
all motors of the Mindstorms NXT.\\

\noindent
{\bf Example}
\begin{lstlisting}
#include <nxt.h> 

ChNXT nxt;

if (!nxt.connectWithAddress("00:16:53:12:e7:80")){
    printf("Connection to NXT has failed.");
    exit(-1);
}
    
nxt.setMotorRelativeZeros();
\end{lstlisting}

\noindent
{\bf Output}\\
None.\\
\\
%%%%% END OF ChNXT::setMotorRelativeZeros %%%%%

\noindent
\vspace{5pt}
\rule{4.5in}{0.015in}\\
\noindent
{\LARGE \texttt{ChNXT::setSensor()} \index{ChNXT::setSensor()}}\\

\addcontentsline{toc}{subsection}{setSensor()}

\noindent
{\bf Synopsis}
\begin{lstlisting}
#include <nxt.h>
int ChNXT::setSensor(nxtSensorId_t id,
                     nxtSensorType type, 
                     nxtSensorMode mode);
\end{lstlisting}

\noindent
{\bf Purpose}\\
Set up a sensor of Lego mindstorms NXT.\\

\noindent
{\bf Return Value}\\
The function returns 0 on success and non-zero otherwise.\\

\noindent
{\bf Parameters}\\
\vspace{-0.1in}
\begin{description}
\item
\begin{tabular}{ p{20mm}p{135mm} }
\texttt{id}&The port of the sensor locate on the Mindstorm NXT.\\
\texttt{type}   &The type of the sensor. \\
\texttt{mode}   &The mode of the sensor. \\
\end{tabular}
\end{description}


\noindent
{\bf NXT Sensor Types}\\
\begin{longtable}{p{5.5cm}p{10cm}}
%\begin{tabular}{p{5.5cm}p{10cm}} \hline
    \hline
Value &       Description\\
\hline
{\tt NXT\_SENSORTYPE\_SWITCH}        &Set to a switch type sensor. Touch sensor is a switch type sensor.\\
%{\tt NXT\_SENSORTYPE\_TEMPERATURE}   &Set to Temperature Sensor.\\
{\tt NXT\_SENSORTYPE\_LIGHT\_ACTIVE} &Set to Light Sensor in light active mode(LED on).\\
{\tt NXT\_SENSORTYPE\_LIGHT\_INACTIVE}&Set to Light Sensor in light inactive mode(LED off).\\
{\tt NXT\_SENSORTYPE\_SOUND\_DB}     &Set to Sound Sensor in dB.\\
{\tt NXT\_SENSORTYPE\_SOUND\_DBA}    &Set to Sound Sensor in dB with adjusted.\\
{\tt NXT\_SENSORTYPE\_LOWSPEED}      &Set to ISP type sensor.\\
{\tt NXT\_SENSORTYPE\_LOWSPEED\_9V}  &Set to ISP type sensor with 9 Voltage. The ultrasonic sensor belongs to this type of sensor.\\
%{\tt NXT\_SENSORTYPE\_HIGHSPEED}     &Not avialable now.\\
{\tt NXT\_SENSORTYPE\_COLORFULL}     &Set to Color Sensor in color detector mode.\\
{\tt NXT\_SENSORTYPE\_COLORRED}      &Set to Color Sensor in lightsensor mode with red light on.\\
{\tt NXT\_SENSORTYPE\_COLORGREEN}    &Set to Color Sensor in lightsensor mode with green light on.\\
{\tt NXT\_SENSORTYPE\_COLORBLUE}     &Set to Color Sensor in lightsensor mode with blue light on.\\
{\tt NXT\_SENSORTYPE\_COLORNONE}     &Set to Color Sensor in lightsensor mode with no light on.\\
\hline
\end{longtable}

\noindent
{\bf NXT Sensor Modes}\\
\begin{longtable}{p{6cm}p{9.5cm}}
%\begin{tabular}{p{6cm}p{9.5cm}} \hline
    \hline
Value & Description\\
\hline
{\tt NXT\_SENSORMODE\_RAWMODE}      &Get sensor value as raw mode.\\	 
{\tt NXT\_SENSORMODE\_BOOLEANMODE}  &Get sensor value as boolean mode.\\
{\tt NXT\_SENSORMODE\_TRANSITIONCNTMODE}  &Get sensor value as a number of transitions between TRUE and FALSE.\\
{\tt NXT\_SENSORMODE\_COUNTERMODE}  &Get sensor value as a number of transitions from FALSE to TRUE, then back to FALSE.\\
{\tt NXT\_SENSORMODE\_PCTFULLSCALEDMODE}  &Get sensor value as percentage of full scale reading for configured sensor type.\\
%{\tt NXT\_SENSORMODE\_CELSIUSMODE}  &Get sensor value of temperature in degrees Celsius.\\
%{\tt NXT\_SENSORMODE\_FAHRENHEITMODE}  &Get sensor value of temperature in degrees Fahrenheit.\\
\hline
%\end{tabular}
\end{longtable}

\noindent
{\bf NXT Sensor Range}\\
\begin{longtable}{p{6cm}p{9.5cm}}
    \hline
    Sensor & Range\\
\hline
Touch Sensor & 0 and 1.\\
Light Sensor & 0 to 100 (percentage).\\
Sound Sensor & 0 to 100 (percentage).\\
Ultrasonic Sensor & 0 to 80 cm.\\
     Color Sensor & 0 to 6.\\
\hline
\end{longtable}

\noindent
{\bf Description}\\
This function is used to set up a sensor on the Mindstorm with a specific type and mode of the sensor. Function {\tt getSensor()} can be used to get the sensor values for each type of sensor.\\

\noindent
{\bf Example}
\begin{lstlisting}
#include <nxt.h> 

ChNXT nxt;
int status_1=2;
int status_2=2;

if (nxt.connectWithAddress("00:16:53:12:e7:80")){
    printf("Connection to NXT has failed.");
    exit(-1);
}
    
/* setup port 1 as touch sensor */
status_1=nxt.setSensor(SENSOR_PORT1, 
        SENSOR_TYPE_TOUCH, SENSOR_MODE_BOOLEANMODE);
if(status_1){
    printf("Connection to touch sensor has failed");
}

/* setup port4 as ultrasonic sensor */
status_2=nxt.setSensor(SENSOR_PORT4, 
        SENSOR_TYPE_ULTRASONIC, SENSOR_MODE_RAWMODE);
if(status_2){
    printf("Connection to ultrasonic sensor has failed");
}
\end{lstlisting}

\noindent
{\bf See Also}\\
{\tt getSensor()}

%%%%% END OF ChNXT::setSensor %%%%%

\noindent
\vspace{5pt}
\rule{4.5in}{0.015in}\\
\noindent
{\LARGE \texttt{ChNXT::setMotorSpeedRatio()} \index{ChNXT::setMotorSpeedRatio()}}\\

\addcontentsline{toc}{subsection}{setMotorSpeedRatio()}

\noindent
{\bf Synopsis}
\begin{lstlisting}
#include <nxt.h>
int ChNXT::setMotorSpeedRatio(nxtMotorId_t id, double ratio);
\end{lstlisting}

\noindent
{\bf Purpose}\\
Set the speed ratio setting of a motor on the Lego Mindstorms NXT.\\

\noindent
{\bf Return Value}\\
The function returns 0 on success and non-zero otherwise.\\

\noindent
{\bf Parameters}\\
\vspace{-0.1in}
\begin{description}
\item
\begin{tabular}{ p{20mm}p{135mm} }
\texttt{id}&The port of the sensor locate on the Mindstorm NXT.\\
\texttt{ratio}&A variable of type double with a value from 0 to 1.\\
\end{tabular}
\end{description}

\noindent
{\bf Description}\\
This function is used to set the speed ratio setting of a motor on
the Lego Mindstorms NXT. The speed ratio setting of a motor is the
percentage of the maximum motor speed, and the value ranges from 
0 to 1. In other words, if the ratio is set to 0.5, the motor will
turn at 50\% of its maximum angular velocity while moving 
continuously or moving to a new goal position.\\

\noindent
{\bf Example}
\begin{lstlisting}
#include <nxt.h> 

ChNXT nxt;

if (nxt.connectWithAddress("00:16:53:12:e7:80")){
    printf("Connection to NXT has failed.");
    exit(-1);
}
    
/* set the speed ratio setting for motor1 on port0 */
nxt.setMotorSpeedRatio(NXT_MOTORA, 0.5);
\end{lstlisting}
\noindent
{\bf See Also}\\
\texttt{setMotorSpeedRatios()}\\
%%%%% END OF ChNXT::connect %%%%%

\noindent
\vspace{5pt}
\rule{4.5in}{0.015in}\\
\noindent
{\LARGE \texttt{ChNXT::setMotorSpeedRatios()} \index{ChNXT::setMotorSpeedRatios()}}\\

\addcontentsline{toc}{subsection}{setMotorSpeedRatios()}

\noindent
{\bf Synopsis}
\begin{lstlisting}
#include <nxt.h>
int ChNXT::setMotorSpeedRatios(double ratioA, double ratioB, double ratioC);
\end{lstlisting}

\noindent
{\bf Purpose}\\
Set the speed ratio setting of all motors on the Lego Mindstorms NXT.\\

\noindent
{\bf Return Value}\\
The function returns 0 on success and non-zero otherwise.\\

\noindent
{\bf Parameters}\\
\vspace{-0.1in}
\begin{description}
\item
\begin{tabular}{ p{20mm}p{135mm} }
\texttt{ratioA}&The speed ratio setting for the first motor.\\
\texttt{ratioB}&The speed ratio setting for the second motor.\\
\texttt{ratioC}&The speed ratio setting for the third motor.\\
\end{tabular}
\end{description}

\noindent
{\bf Description}\\
This function is used to simultaneously set the speed ratio settings 
of all motors on the Lego Mindstorms NXT. The speed ratio setting of
a motor is the percentage of the maximum motor speed, and the value 
ranges from 0 to 1.\\

\noindent
{\bf Example}
\begin{lstlisting}
#include <nxt.h> 

ChNXT nxt;

if (nxt.connectWithAddress("00:16:53:12:e7:80")){
    printf("Connection to NXT has failed.");
    exit(-1);
}
    
/* set the speed ratio settings for all motors*/
nxt.setMotorSpeedRatios(0.5, 0.4, 0.4);
\end{lstlisting}

\noindent
{\bf See Also}\\
\texttt{setMotorSpeedRatio()}\\
%%%%% END OF ChNXT::connect %%%%%

\noindent
\vspace{5pt}
\rule{4.5in}{0.015in}\\
\noindent
{\LARGE \texttt{ChNXT::stopOneMotor()} \index{ChNXT::stopOneMotor()}}\\

\addcontentsline{toc}{subsection}{stopOneMotor()}

\noindent
{\bf Synopsis}
\begin{lstlisting}
#include <nxt.h>
int ChNXT::stopOneMotor(nxtMotorId_t id);
\end{lstlisting}

\noindent
{\bf Purpose}\\
Stop a motor moving.\\

\noindent
{\bf Return Value}\\
The function returns 0 on success and non-zero otherwise.\\

\noindent
{\bf Parameters}\\
\vspace{-0.1in}
\begin{description}
\item
\begin{tabular}{ p{20mm}p{135mm} }
\texttt{id}       &The motor of the Mindstorm NXT.\\
\end{tabular}
\end{description}

\noindent
{\bf Description}\\
This function is used to stop the specific motor on the Mindstorm NXT.\\

\noindent
{\bf Example}
\begin{lstlisting}
#include <nxt.h> 

ChNXT nxt;

if (nxt.connectWithAddress("00:16:53:12:e7:80")){
    printf("Connection to NXT has failed.");
    exit(-1);
}
    
/* setup speed for all three motors */
nxt.setMotorSpeeds(60, 40, 40);

/* move motor1 360 degrees */
nxt.moveMotorContinuousNB(NXT_MOTORA, NXT_FORWARD);
delay(5);

/* stop motor1 */
nxt.stopOneMotor(NXT_MOTORA);
\end{lstlisting}

\noindent
{\bf See Also}\\
\texttt{stopTwoMotors(), stopAllMotors()}\\
%%%%% END OF ChNXT::stopOneMotor() %%%%%

\noindent
\vspace{5pt}
\rule{4.5in}{0.015in}\\
\noindent
{\LARGE \texttt{ChNXT::stopTwoMotors()} \index{ChNXT::stopTwoMotors()}}\\

\addcontentsline{toc}{subsection}{stopTwoMotors()}

\noindent
{\bf Synopsis}
\begin{lstlisting}
#include <nxt.h>
int ChNXT::stopTwoMotors(nxtMotorId_t id1, nxtMotorId_t id2);
\end{lstlisting}

\noindent
{\bf Purpose}\\
Stop two motors moving.\\

\noindent
{\bf Return Value}\\
The function returns 0 on success and non-zero otherwise.\\

\noindent
{\bf Parameters}\\
\vspace{-0.1in}
\begin{description}
\item
\begin{tabular}{ p{20mm}p{135mm} }
\texttt{id1}       &A motor of the Mindstorm NXT.\\
\texttt{id2}      &Another motor of the Mindstorm NXT.\\
\end{tabular}
\end{description}

\noindent
{\bf Description}\\
This function is used to stop two specific motors on the Mindstorm NXT.\\

\noindent
{\bf Example}
\begin{lstlisting}
#include <nxt.h> 

ChNXT nxt;

if (nxt.connectWithAddress("00:16:53:12:e7:80")){
    printf("Connection to NXT has failed.");
    exit(-1);
}
    
/* setup speed for all three motors */
nxt.setMotorSpeeds(60, 40, 40);

/* move motor2 and motor3 forward */
nxt.moveMotorContinuousNB(NXT_MOTORB, NXT_FORWARD);
nxt.moveMotorContinuousNB(NXT_MOTORC, NXT_FORWARD);
delay(5);

/* stop motor2 and motor3 */
nxt.stopTwoMotors(NXT_MOTORB, NXT_MOTORC);
\end{lstlisting}

\noindent
{\bf See Also}\\
\texttt{stopOneMotor(), stopAllMotors()}\\
%%%%% END OF ChNXT::stopOneMotor( ) %%%%%

\noindent
\vspace{5pt}
\rule{4.5in}{0.015in}\\
\noindent
{\LARGE \texttt{ChNXT::stopAllMotors()} \index{ChNXT::stopAllMotors()}}\\

\addcontentsline{toc}{subsection}{stopAllMotors()}

\noindent
{\bf Synopsis}
\begin{lstlisting}
#include <nxt.h>
int ChNXT::stopAllMotors(void);
\end{lstlisting}

\noindent
{\bf Purpose}\\
Stop all motors moving.\\

\noindent
{\bf Return Value}\\
The function returns 0 on success and non-zero otherwise.\\

\noindent
{\bf Parameters}\\
None.\\

\noindent
{\bf Description}\\
This function is used to stop all motors on the Mindstorm NXT.\\

\noindent
{\bf Example}
\begin{lstlisting}
#include <nxt.h> 

ChNXT nxt;

if (nxt.connectWithAddress("00:16:53:12:e7:80")){
    printf("Connection to NXT has failed.");
    exit(-1);
}
    
/* setup speed for all three motors */
nxt.setMotorSpeeds(60, 40, 40);

/* move all motors forward */
nxt.moveMotorContinuousNB(NXT_MOTORA, NXT_FORWARD);
nxt.moveMotorContinuousNB(NXT_MOTORB, NXT_FORWARD);
nxt.moveMotorContinuousNB(NXT_MOTORC, NXT_FORWARD);
delay(5);

/* stop all motors */
nxt.stopAllMotors();
\end{lstlisting}

\noindent
{\bf See Also}\\
\texttt{stopOneMotor(), stopTwoMotors()}\\
%%%%% END OF ChNXT::stopAllMotors %%%%%

\noindent
\vspace{5pt}
\rule{4.5in}{0.015in}\\
\noindent
{\LARGE \texttt{ChNXT::vehicleRollBackward()}\index{ChNXT::vehicleRollBackward()}}\\
{\LARGE \texttt{ChNXT::vehicleRollBackwardNB()}\index{ChNXT::vehicleRollBackwardNB()}}\\
%\phantomsection
\addcontentsline{toc}{subsection}{vehicleRollBackward()}
\addcontentsline{toc}{subsection}{vehicleRollBackwardNB()}

\noindent
{\bf Synopsis}
\vspace{-8pt}
\begin{verbatim}
#include <nxt.h>
int ChNXT::vehicleRollBackward(double angle);
int ChNXT::vehicleRollBackwardNB(double angle);
\end{verbatim}

\noindent
{\bf Purpose}\\
Make the Lego Mindstorms NXT in the vehicle configuration roll backward.\\

\noindent
{\bf Return Value}\\
The function returns 0 on success and non-zero otherwise.\\

\noindent
{\bf Parameters}\\
\vspace{-0.1in}
\begin{description}
\item               
\begin{tabular}{p{15 mm}p{145 mm}}
\texttt{angle} & The angle to turn the wheels, specified in degrees.\\
\end{tabular}
\end{description}

\noindent
{\bf Description}\\
\vspace{-12pt}
\begin{quote}
{\bf ChNXT::vehicleRollBackward()}\\
This function is used to roll the Lego Mindstorms NXT backward in the vehicle
configuration. The amount to roll the wheels is specified by the argument,
\texttt{angle}.

{\bf ChNXT::vehicleRollBackwardNB()}\\
This function is used to roll the Lego Mindstorms NXT backward in the vehicle
configuration. The amount to roll the wheels is specified by the argument,
\texttt{angle}.

This function has both a blocking and non-blocking version.
The blocking version, \texttt{vehicleRollBackward()}, will block until the
robot motion has completed. The non-blocking version, \texttt{vehicleRollBackwardNB()},
will return immediately, and the motion will be performed asynchronously.\\
\end{quote}

\noindent
{\bf Example}
\begin{verbatim}
#include <nxt.h>

ChNXT nxt;

nxt.connect();

/* Blocking function */
nxt.vehicleRollBackward(360);

/* Non-blocking function */
nxt.vehicleRollBackwardNB(360);
nxt.vehicleMotionWait();

nxt.disconnect();
\end{verbatim}

\noindent
{\bf See Also}\\
\texttt{vehicleRollForward()}

%\CPlot::\DataThreeD(), \CPlot::\DataFile(), \CPlot::\Plotting(), \plotxy().\\

\noindent
\vspace{5pt}
\rule{4.5in}{0.015in}\\
\noindent
{\LARGE \texttt{ChNXT::vehicleRollForward()}\index{ChNXT::vehicleRollForward()}}\\
{\LARGE \texttt{ChNXT::vehicleRollForwardNB()}\index{ChNXT::vehicleRollForwardNB()}}\\
%\phantomsection
\addcontentsline{toc}{subsection}{vehicleRollForward()}
\addcontentsline{toc}{subsection}{vehicleRollForwardNB()}

\noindent
{\bf Synopsis}
\vspace{-8pt}
\begin{verbatim}
#include <mobot.h>
int ChNXT::vehicleRollForward(double angle);
int ChNXT::vehicleRollForwardNB(double angle);
\end{verbatim}

\noindent
{\bf Purpose}\\
Make the Lego Mindstorms NXT in the vehicle configuration roll forward.\\

\noindent
{\bf Return Value}\\
The function returns 0 on success and non-zero otherwise.\\

\noindent
{\bf Parameters}\\
\vspace{-0.1in}
\begin{description}
\item               
\begin{tabular}{p{15 mm}p{145 mm}}
\texttt{angle} & The angle to turn the wheels, specified in degrees.\\
\end{tabular}
\end{description}

\noindent
{\bf Description}\\
\vspace{-12pt}
\begin{quote}
{\bf ChNXT::vehicleRollForward()}\\
This function is used to roll the Lego Mindstorms forward in the vehicle
configuration. The amount to roll the wheels is specified by the argument,
\texttt{angle}.

{\bf ChNXT::vehicleRollForwardNB()}\\
This function is used to roll the Lego Mindstorms forward in the vehicle
configuration. The amount to roll the wheels is specified by the argument,
\texttt{angle}.

This function has both a blocking and non-blocking version.
The blocking version, \texttt{vehicleRollForward()}, will block until the
robot motion has completed. The non-blocking version, \texttt{vehicleRollForwardNB()},
will return immediately, and the motion will be performed asynchronously.\\
\end{quote}

\noindent
{\bf Example}
\begin{verbatim}
#include <nxt.h>

ChNXT nxt;

nxt.connect();

/* Blocking function */
nxt.vehicleRollForward(360);

/* Non-blocking function */
nxt.vehicleRollForwardNB(360);
nxt.vehicleMotionWait();

nxt.disconnect();
\end{verbatim}

\noindent
{\bf See Also}\\
\texttt{vehicleRollBackward()}

%\CPlot::\DataThreeD(), \CPlot::\DataFile(), \CPlot::\Plotting(), \plotxy().\\

\noindent
\vspace{5pt}
\rule{4.5in}{0.015in}\\
\noindent
{\LARGE \texttt{ChNXT::vehicleRotateLeft()}\index{ChNXT::vehicleRotateLeft()}}\\
{\LARGE \texttt{ChNXT::vehicleRotateLeftNB()}\index{ChNXT::vehicleRotateLeftNB()}}\\
%\phantomsection
\addcontentsline{toc}{subsection}{vehicleRotateLeft()}
\addcontentsline{toc}{subsection}{vehicleRotateLeftNB()}

\noindent
{\bf Synopsis}
\vspace{-8pt}
\begin{verbatim}
#include <nxt.h>
int ChNXT::vehicleRotateLeft(double angle);
int ChNXT::vehicleRotateLeftNB(double angle);
\end{verbatim}

\noindent
{\bf Purpose}\\
Rotate the Lego Mindstorms NXT left in vehicle configuration.\\

\noindent
{\bf Return Value}\\
The function returns 0 on success and non-zero otherwise.\\

\noindent
{\bf Parameters}\\
\vspace{-0.1in}
\begin{description}
\item               
\begin{tabular}{p{10 mm}p{145 mm}}
\texttt{angle} & The angle in degrees to turn the wheels. The wheels will turn 
    in opposite directions by the amount specifid by this argument in order to 
    rotate the robot to the left. \\
\end{tabular}
\end{description}

\noindent
{\bf Description}\\
\vspace{-12pt}
\begin{quote}
{\bf ChNXT::vehicleRotateLeft()}\\
This function is used to rotate the wheels of the Lego Mindstorms NXT in vehicle configuration in opposite directions to cause the robot to rotate counter-clockwise.

{\bf ChNXT::vehicleRotateLeftNB()}\\
This function is used to rotate the wheels of the Lego Mindstorms NXT in vehicle configuration in opposite directions to cause the robot to rotate counter-clockwise.\\
\newline
This function has both a blocking and non-blocking version.
The blocking version, \texttt{vehicleRotateLeft()}, will block until the
robot motion has completed. The non-blocking version, \texttt{vehicleRotateLeftNB()},
will return immediately, and the motion will be performed asynchronously.\\
\end{quote}

\noindent
{\bf Example}
\begin{verbatim}
#include <nxt.h>

ChNXT nxt;

nxt.connect();

/* Blocking function */
nxt.vehicleRotateLeft(360);

/* Non-blocking function */
nxt.vehicleRotateLeftNB(360);
nxt.vehicleMotionWait();

nxt.disconnect();
\end{verbatim}

\noindent
{\bf See Also}\\
\texttt{vehicleRotateRight()}

%\CPlot::\DataThreeD(), \CPlot::\DataFile(), \CPlot::\Plotting(), \plotxy().\\

\noindent
\vspace{5pt}
\rule{4.5in}{0.015in}\\
\noindent
{\LARGE \texttt{ChNXT::vehicleRotateRight()}\index{ChNXT::vehicleRotateRight()}}\\
{\LARGE \texttt{ChNXT::vehicleRotateRightNB()}\index{ChNXT::vehicleRotateRightNB()}}\\
%\phantomsection
\addcontentsline{toc}{subsection}{vehicleRotateRight()}
\addcontentsline{toc}{subsection}{vehicleRotateRightNB()}

\noindent
{\bf Synopsis}
\vspace{-8pt}
\begin{verbatim}
#include <mobot.h>
int ChNXT::vehicleRotateRight(double angle);
int ChNXT::vehicleRotateRightNB(double angle);
\end{verbatim}

\noindent
{\bf Purpose}\\
Rotate the Lego Mindstorms NXT right in vehicle configuration.\\

\noindent
{\bf Return Value}\\
The function returns 0 on success and non-zero otherwise.\\

\noindent
{\bf Parameters}\\
\vspace{-0.1in}
\begin{description}
\item               
\begin{tabular}{p{10 mm}p{145 mm}}
\texttt{angle} & The angle in degrees to turn the wheels. The wheels will turn in opposite directions by the amount specifid by this argument in order to turn the robot to the right. \\
\end{tabular}
\end{description}

\noindent
{\bf Description}\\
\vspace{-12pt}
\begin{quote}
{\bf ChNXT::vehicleRotateRight()}\\
This function causes the robot to rotate the wheels of the Lego Mindstorms NXT in vehicle configuration in opposite directions to cause the robot to rotate clockwise.

{\bf ChNXT::vehicleRotateRightNB()}\\
This function causes the robot to rotate the wheels of the Lego Mindstorms NXT in vehicle configuration in opposite directions to cause the robot to rotate clockwise.\\
\newline
This function has both a blocking and non-blocking version.
The blocking version, \texttt{vehicleRotateRight()}, will block until the
robot motion has completed. The non-blocking version, \texttt{vehicleRotateRightNB()},
will return immediately, and the motion will be performed asynchronously.\\
\end{quote}

\noindent
{\bf Example}
\begin{verbatim}
#include <nxt.h>

ChNXT nxt;

nxt.connect();

/* Blocking function */
nxt.vehicleRotateRight(360);

/* Non-blocking function */
nxt.vehicleRotateRightNB(360);
nxt.vehicleMotionWait();

nxt.disconnect();
\end{verbatim}

\noindent
{\bf See Also}\\
\texttt{vehicleRotateRight()}

%\CPlot::\DataThreeD(), \CPlot::\DataFile(), \CPlot::\Plotting(), \plotxy().\\

\noindent
\vspace{5pt}
\rule{4.5in}{0.015in}\\
\noindent
{\LARGE \texttt{ChNXT::vehicleMotionWait()}\index{ChNXT::vehicleMotionWait()}}\\
%\phantomsection
\addcontentsline{toc}{subsection}{vehicleMotionWait()}

\noindent
{\bf Synopsis}
\vspace{-8pt}
\begin{verbatim}
#include <nxt.h>
int ChNXT::vehicleMotionWait();
\end{verbatim}

\noindent
{\bf Purpose}\\
Wait for a motion to complete execution in vehicle configuration.\\

\noindent
{\bf Return Value}\\
The function returns 0 on success and non-zero otherwise.\\

\noindent
{\bf Description}\\
This function is used to wait for a motion function to fully complete its cycle.\\


\noindent
{\bf Example}\\
\begin{verbatim}
#include <nxt.h>

ChNXT nxt;

nxt.connect();

/* Non-blocking function */
nxt.vehicleRotateRightNB(360);

/* wait until non-blocking motion stops */
nxt.vehicleMotionWait();
\end{verbatim}

\noindent
{\bf See Also}\\
\texttt{vehicleRotateRight()}, \texttt{vehicleRotateLeft()}, \\
\texttt{vehicleRollForward()}, \texttt{vehicleRollBackward()}



%\CPlot::\DataThreeD(), \CPlot::\DataFile(), \CPlot::\Plotting(), \plotxy().\\


%%%%%%%%%%%%%%%%%%%%%%%%%%%%%%%%%%% End of Appendix chnxt_api %%%%%%%%%%%%%%%%%%%%%%%%%%%%%%%%%%%%%%

%%%%%%%%%%%%%%%%%%%%%%%%%%%%%%%%%%% Appendix ultility_functions %%%%%%%%%%%%%%%%%%%%%%%%%%%%%%%%%%%%%
\clearpage
\newpage
\section{\label{sec:ultility_functions}Miscellaneous Utility Functions}
Besides the control functions described in Appendix \ref{sec:chnxt_api}, some utility functions are
also very useful when you are programming, such as convert a certain angle to distance with the 
coresponding radius of wheel. Therefore, we will introduce you some useful utility functions in 
this section.\\
%\lhead{ChNXT API Documentation}
%\noindent
%There are several utility functions which are useful when programming for
%the NXT. 
\subsection{Overview}
\begin{table}[H]
%\capstart
\begin{center}
\caption{NXT Utility Functions.}
\begin{tabular}{p{38 mm}p{110 mm}}
%\begin{tabular}{ll}
\hline
Function & Description \\
\hline
%\texttt{pose()} \dotfill & Pose multiple motors of the robot. \\
\texttt{angle2distance()} & Calculates the angle a wheel has turned from the
radius and distance traveled.\\
\texttt{deg2rad()} & Converts degrees to radians. \\
\texttt{delay()} & Puts a delay into a program. \\
\texttt{distance2angle()} & Calculates the distance traveled by a wheel from the wheel's radius and angle turned.\\
\texttt{rad2deg()} & Converts radians to degrees.\\
%\texttt{shiftTime()} & Shift the data of a plot.\\
\hline
\end{tabular}
\end{center}
\label{mobilec_api_cbinary}
\end{table}

\subsection{Function Details}
\noindent
\vspace{5pt}
\rule{4.5in}{0.015in}\\
\noindent
{\LARGE \texttt{angle2distance()}\index{angle2distance()}}\\
%\phantomsection
\addcontentsline{toc}{subsection}{angle2distance()}

\noindent
{\bf Synopsis}
\vspace{-8pt}
\begin{verbatim}
#include <mobot.h>
double angle2distance(double radius, double angle);
array double angle2distance(double radius, array double angle[:])[:];
\end{verbatim}

\noindent
{\bf Purpose}\\
Calculate the distance a wheel has traveled from the radius of the wheel and
the angle the wheel has turned.\\

\noindent
{\bf Return Value}\\
The value returned is the distance traveled by the wheel. If the angle argument is an
array of angles, then the value returned is an array of distances. Each element
of the distance array returned is the distance calculated from the respective
element in the angle array.\\

\noindent
{\bf Parameters}
\vspace{-0.1in}
\begin{description}
\item               
\begin{tabular}{p{15 mm}p{145 mm}}
\texttt{radius} & The radius of the wheel. \\
\texttt{angle} & This value is the angle the wheel has turned. This parameter may be of \texttt{double} type, or a Ch computational array. \\
\end{tabular}
\end{description}

\noindent
{\bf Description}\\
This function calculates the angle a wheel has turned given the wheel 
radius and distance traveled. The equation used is
\begin{equation*}
d = r \theta
\end{equation*}
where $d$ is the distance traveled, $r$ is the radius of the wheel, and $\theta$ is
the angle the wheel has turned in radians.
\\

\noindent
{\bf Example}\\
\noindent

\noindent
{\bf See Also}\\
\texttt{distance2angle()}

%\CPlot::\DataThreeD(), \CPlot::\DataFile(), \CPlot::\Plotting(), \plotxy().\\

\noindent
\vspace{5pt}
\rule{4.5in}{0.015in}\\
\noindent
{\LARGE \texttt{deg2rad()}\index{deg2rad()}}\\
%\phantomsection
\addcontentsline{toc}{subsection}{deg2rad()}

\noindent
{\bf Synopsis}
\vspace{-8pt}
\begin{verbatim}
#include <nxt.h>
double deg2rad(double degrees);
array double deg2rad(double degrees[:])[:];
\end{verbatim}

\noindent
{\bf Purpose}\\
Convert degrees to radians.\\

\noindent
{\bf Return Value}\\
The angle parameter converted to radians.\\

\noindent
{\bf Parameters}
\vspace{-0.1in}
\begin{description}
\item               
\begin{tabular}{p{15 mm}p{145 mm}}
\texttt{degrees} & The angle to convert, in degrees. \\
\end{tabular}
\end{description}

\noindent
{\bf Description}\\
This function converts an angle expressed in degrees into radians. Degrees and
radians are two popular ways to express an angle, though they are not interchangable.
The following equation is used to convert degrees to radians:
\begin{equation*}
\theta = \delta * \frac{\pi}{180}
\end{equation*}
where $\theta$ is the angle in radians and $\delta$ is the angle in degrees.

\noindent
{\bf Example}\\
\noindent

\noindent
{\bf See Also}\\
\texttt{rad2deg()}

%\CPlot::\DataThreeD(), \CPlot::\DataFile(), \CPlot::\Plotting(), \plotxy().\\

\noindent
\vspace{5pt}
\rule{4.5in}{0.015in}\\
\noindent
{\LARGE \texttt{delay()}\index{delay()}}\\
%\phantomsection
\addcontentsline{toc}{subsection}{delay()}

\noindent
{\bf Synopsis}
\vspace{-8pt}
\begin{verbatim}
#include <mobot.h>
void delay(double seconds);
\end{verbatim}

\noindent
{\bf Purpose}\\
Pause a program for a set amount of time.\\

\noindent
{\bf Return Value}\\
None.\\

\noindent
{\bf Parameters}
\vspace{-0.1in}
\begin{description}
\item               
\begin{tabular}{p{15 mm}p{145 mm}}
\texttt{seconds} & The number of seconds to delay. \\
\end{tabular}
\end{description}

\noindent
{\bf Description}\\
This function delays or pauses a program for a number of seconds. For instance, 
the code 
\begin{verbatim}
delay(0.5);
printf("Hello.\n");
delay(2);
printf("Goodbye.\n");
\end{verbatim}
will pause for half a second, print the text \texttt{Hello.}, delay for 2 seconds,
and then print the text \texttt{Goodbye.}. 
\noindent
{\bf Example}\\
\noindent

\noindent
{\bf See Also}\\

%\CPlot::\DataThreeD(), \CPlot::\DataFile(), \CPlot::\Plotting(), \plotxy().\\

%\noindent
\vspace{5pt}
\rule{4.5in}{0.015in}\\
\noindent
{\LARGE \texttt{pause()}\index{pause()}}\\
%\phantomsection
\addcontentsline{toc}{subsection}{pause()}

\noindent
{\bf Synopsis}
\vspace{-8pt}
\begin{verbatim}
#include <mobot.h>
void pause(double seconds);
\end{verbatim}

\noindent
{\bf Purpose}\\
Pause a program for a set amount of time.\\

\noindent
{\bf Return Value}\\
None.\\

\noindent
{\bf Parameters}
\vspace{-0.1in}
\begin{description}
\item               
\begin{tabular}{p{15 mm}p{145 mm}}
\texttt{seconds} & The number of seconds to pause. \\
\end{tabular}
\end{description}

\noindent
{\bf Description}\\
This function delays or pauses a program for a number of seconds. For instance, 
the code 
\begin{verbatim}
pause(0.5);
printf("Hello.\n");
pause(2);
printf("Goodbye.\n");
\end{verbatim}
will pause for half a second, print the text \texttt{Hello.}, pause for 2 seconds,
and then print the text \texttt{Goodbye.}. 
\noindent
{\bf Example}\\
%\noindent

%\noindent
%{\bf See Also}\\

%\CPlot::\DataThreeD(), \CPlot::\DataFile(), \CPlot::\Plotting(), \plotxy().\\

\noindent
\vspace{5pt}
\rule{4.5in}{0.015in}\\
\noindent
{\LARGE \texttt{distance2angle()}\index{distance2angle()}}\\
%\phantomsection
\addcontentsline{toc}{subsection}{distance2angle()}

\noindent
{\bf Synopsis}
\vspace{-8pt}
\begin{verbatim}
#include <mobot.h>
double distance2angle(double radius, double distance);
array double distance2angle(double radius, array double distance[:])[:];
\end{verbatim}

\noindent
{\bf Purpose}\\
Calculate the angle a wheel has turned from the radius of the wheel and
the distance the wheel has traveled.\\

\noindent
{\bf Return Value}\\
The value returned is the angle turned by the wheel in degrees. If the distance argument is an
array of distances, then the value returned is an array of angles. Each element
of the angle array returned is the angle calculated from the respective
element in the distance array.\\

\noindent
{\bf Parameters}
\vspace{-0.1in}
\begin{description}
\item               
\begin{tabular}{p{15 mm}p{145 mm}}
\texttt{radius} & The radius of the wheel. \\
\texttt{distance} & This value is the distance the wheel has traveled. This parameter may be of \texttt{double} type, or a Ch computational array. \\
\end{tabular}
\end{description}

\noindent
{\bf Description}\\
This function calculates the distance a wheel has turned given the wheel 
radius and angle turned. The equation used is
\begin{equation*}
\theta = \frac{d}{r}
\end{equation*}
where $d$ is the distance traveled, $r$ is the radius of the wheel, and $\theta$ is
the angle the wheel has turned in radians. A further conversion is done in the code to
convert the angle from radians into degrees before returning the value.
\\

\noindent
{\bf Example}\\
\noindent

\noindent
{\bf See Also}\\
\texttt{angle2distance()}

%\CPlot::\DataThreeD(), \CPlot::\DataFile(), \CPlot::\Plotting(), \plotxy().\\

\clearpage
\newpage
\noindent
\vspace{5pt}
\rule{4.5in}{0.015in}\\
\noindent
{\LARGE \texttt{rad2deg()}\index{rad2deg()}}\\
%\phantomsection
\addcontentsline{toc}{subsection}{rad2deg()}

\noindent
{\bf Synopsis}
\vspace{-8pt}
\begin{verbatim}
#include <nxt.h>
double rad2deg(double radians);
array double rad2deg(double radians[:])[:];
\end{verbatim}

\noindent
{\bf Purpose}\\
Convert radians to degrees.\\

\noindent
{\bf Return Value}\\
The angle parameter converted to degrees.\\

\noindent
{\bf Parameters}
\vspace{-0.1in}
\begin{description}
\item               
\begin{tabular}{p{15 mm}p{145 mm}}
\texttt{radians} & The angle to convert, in radians. \\
\end{tabular}
\end{description}

\noindent
{\bf Description}\\
This function converts an angle expressed in radians into degrees. Degrees and
radians are two popular ways to express an angle, though they are not interchangable.
The following equation is used to convert radians to degrees:
\begin{equation*}
\delta = \theta * \frac{180}{\pi}
\end{equation*}
where $\theta$ is the angle in radians and $\delta$ is the angle in degrees.

\noindent
{\bf Example}\\
\noindent

\noindent
{\bf See Also}\\
\texttt{deg2rad()}

%\CPlot::\DataThreeD(), \CPlot::\DataFile(), \CPlot::\Plotting(), \plotxy().\\

%\noindent
\vspace{5pt}
\rule{4.5in}{0.015in}\\
\noindent
{\LARGE \texttt{shiftTime()}\index{shiftTime()}}\\
%\phantomsection
\addcontentsline{toc}{subsection}{shiftTime()}

\noindent
{\bf Synopsis}
\vspace{-8pt}
\begin{verbatim}
#include <mobot.h>
int shiftTime(double tolerance, int numDataPoints, double time[], double data1[], ...);
\end{verbatim}

\noindent
{\bf Syntax}
\vspace{-8pt}
\begin{verbatim}
#include <mobot.h>
shiftTime(tolerance, numDataPoints, time, angles1);
shiftTime(tolerance, numDataPoints, time, angles1, angles2);
shiftTime(tolerance, numDataPoints, time, angles1, angles2, angles3);
shiftTime(tolerance, numDataPoints, time, angles1, angles2, angles3, angles4);
etc...
\end{verbatim}

\noindent
{\bf Purpose}\\
This function is used to shift the data in one or more plots to the left. It is commonly used to
line up the beginning of robot motions with the y-axis on plots.\\

\noindent
{\bf Return Value}\\
The return value is the number of elements which have been shifted of the plots. \\

\noindent
{\bf Parameters}
\vspace{-0.1in}
\begin{description}
\item               
\begin{tabular}{p{25 mm}p{145 mm}}
\texttt{tolerance} & The angle tolerance to detect the beginning of the motion. A lower tolerance
is more sensitive to small motions, but also more sensitive to noise. A higher tolerance will
reject noise, but may yield an inaccurate shift in time such that the motion does not appear to
begin at time 0. \\
\texttt{numDataPoints} & The number of elements in the arrays. \\
\texttt{time} & The array holding time or "x-axis" values. \\
\texttt{data1} & An array holding data. \\
\texttt{...} & Additional arrays holding data. 
\end{tabular}
\end{description}

\noindent
{\bf Description}\\
This function is used to shift data to the left to align motion start times with the
y-axis. This is done by detecting a change in value in any of the data arrays provided
to the function. If there is a change greater than the value provided as the
tolerance, that time is labeled as the beginning of the motion. All data points
prior to the beginning of the motion are deleted, and the beginning of the
motion is aligned with time 0.

\noindent
{\bf Example}\\
\noindent

\noindent
{\bf See Also}\\

%\CPlot::\DataThreeD(), \CPlot::\DataFile(), \CPlot::\Plotting(), \plotxy().\\


%%%%%%%%%%%%%%%%%%%%%%%%%%%%%%%%%% End of appendix ultility_functions %%%%%%%%%%%%%%%%%%%%%%%%%%%%%%
