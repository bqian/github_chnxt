\noindent
\vspace{5pt}
\rule{4.5in}{0.015in}\\
\noindent
{\LARGE \texttt{pause()}\index{pause()}}\\
%\phantomsection
\addcontentsline{toc}{subsection}{pause()}

\noindent
{\bf Synopsis}
\vspace{-8pt}
\begin{verbatim}
#include <mobot.h>
void pause(double seconds);
\end{verbatim}

\noindent
{\bf Purpose}\\
Pause a program for a set amount of time.\\

\noindent
{\bf Return Value}\\
None.\\

\noindent
{\bf Parameters}
\vspace{-0.1in}
\begin{description}
\item               
\begin{tabular}{p{15 mm}p{145 mm}}
\texttt{seconds} & The number of seconds to pause. \\
\end{tabular}
\end{description}

\noindent
{\bf Description}\\
This function delays or pauses a program for a number of seconds. For instance, 
the code 
\begin{verbatim}
pause(0.5);
printf("Hello.\n");
pause(2);
printf("Goodbye.\n");
\end{verbatim}
will pause for half a second, print the text \texttt{Hello.}, pause for 2 seconds,
and then print the text \texttt{Goodbye.}. 
\noindent
{\bf Example}\\
%\noindent

%\noindent
%{\bf See Also}\\

%\CPlot::\DataThreeD(), \CPlot::\DataFile(), \CPlot::\Plotting(), \plotxy().\\
