\noindent
\vspace{5pt}
\rule{4.5in}{0.015in}\\
\noindent
{\LARGE \texttt{distance2angle()}\index{distance2angle()}}\\
%\phantomsection
\addcontentsline{toc}{subsection}{distance2angle()}

\noindent
{\bf Synopsis}
\vspace{-8pt}
\begin{verbatim}
#include <nxt.h>
double distance2angle(double radius, double distance);
array double distance2angle(double radius, array double distance[:])[:];
\end{verbatim}

\noindent
{\bf Purpose}\\
Calculate the angle a wheel has turned from the radius of the wheel and
the distance the wheel has traveled.\\

\noindent
{\bf Return Value}\\
The value returned is the angle turned by the wheel in degrees. If the distance argument is an
array of distances, then the value returned is an array of angles. Each element
of the angle array returned is the angle calculated from the respective
element in the distance array.\\

\noindent
{\bf Parameters}
\vspace{-0.1in}
\begin{description}
\item               
\begin{tabular}{p{15 mm}p{145 mm}}
\texttt{radius} & The radius of the wheel. \\
\texttt{distance} & This value is the distance the wheel has traveled. This parameter may be of \texttt{double} type, or a Ch computational array. \\
\end{tabular}
\end{description}

\noindent
{\bf Description}\\
This function calculates the distance a wheel has turned given the wheel 
radius and angle turned. The equation used is
\begin{equation*}
\theta = \frac{d}{r}
\end{equation*}
where $d$ is the distance traveled, $r$ is the radius of the wheel, and $\theta$ is
the angle the wheel has turned in radians. A further conversion is done in the code to
convert the angle from radians into degrees before returning the value.
\\

\noindent
{\bf Example}\\
\noindent

\noindent
{\bf See Also}\\
\texttt{angle2distance()}

%\CPlot::\DataThreeD(), \CPlot::\DataFile(), \CPlot::\Plotting(), \plotxy().\\
