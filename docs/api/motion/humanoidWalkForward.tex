\noindent
\vspace{5pt}
\rule{4.5in}{0.015in}\\
\noindent
{\LARGE \texttt{ChNXT::humanoidWalkForward()}\index{ChNXT::humanoidWalkForward()}}\\
{\LARGE \texttt{ChNXT::humanoidWalkForwardNB()}\index{ChNXT::humanoidWalkForwardNB()}}\\
%\phantomsection
\addcontentsline{toc}{subsection}{humanoidWalkForward()}
\addcontentsline{toc}{subsection}{humanoidWalkForwardNB()}

\noindent
{\bf Synopsis}
\begin{lstlisting}
#include <mobot.h>
int ChNXT::humanoidWalkForward(double angle);
int ChNXT::humanoidWalkForwardNB(double angle);
\end{lstlisting}

\noindent
{\bf Purpose}\\
Make the Lego Mindstorms NXT in the humanoid configuration walk forward.\\

\noindent
{\bf Return Value}\\
The function returns 0 on success and non-zero otherwise.\\

\noindent
{\bf Parameters}\\
\vspace{-0.1in}
\begin{description}
\item               
\begin{tabular}{p{15 mm}p{145 mm}}
\texttt{angle} & The angle to turn the joints, specified in degrees.\\
\end{tabular}
\end{description}

\noindent
{\bf Description}\\
\vspace{-12pt}
\begin{quote}
{\bf ChNXT::humanoidWalkForward()}\\
This function is used to make the Lego Mindstorms walk forward in the humanoid
configuration. The amount to roll the joints is specified by the argument,
\texttt{angle}.

{\bf ChNXT::humanoidWalkForwardNB()}\\
This function is used to make the Lego Mindstorms walk forward in the humanoid
configuration. The amount to roll the joints is specified by the argument,
\texttt{angle}.

This function has both a blocking and non-blocking version.
The blocking version, \texttt{humanoidWalkForward()}, will block until the
robot motion has completed. The non-blocking version, \texttt{humanoidWalkForwardNB()},
will return immediately, and the motion will be performed asynchronously.\\
\end{quote}

\noindent
{\bf Example}
\begin{lstlisting}
#include <nxt.h>

ChNXT nxt;

nxt.connect();

/* Blocking function */
nxt.humanoidWalkForward(360);

/* Non-blocking function */
nxt.humanoidWalkForwardNB(360);
nxt.humanoidMotionWait();
\end{lstlisting}

\noindent
{\bf See Also}\\
\texttt{humanoidWalkBackward()}

%\CPlot::\DataThreeD(), \CPlot::\DataFile(), \CPlot::\Plotting(), \plotxy().\\
