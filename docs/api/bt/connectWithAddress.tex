\noindent
\vspace{5pt}
\rule{4.5in}{0.015in}\\
\noindent
{\LARGE \texttt{ChNXT::connectWithAddress()} \index{ChNXT::connectWithAddress()}}\\

\addcontentsline{toc}{subsection}{connectWithAddress()}

\noindent
{\bf Synopsis}
\vspace{-8pt}
\begin{verbatim}
#include <nxt.h>
int ChNXT::connectWithAddress(char usr_addr[18]);
\end{verbatim}

\noindent
{\bf Purpose}\\
Connect to a Mindstorms NXT via Bluetooth.\\

\noindent
{\bf Return Value}\\
The function returns 0 on success, and non-zero otherwise.\\

\noindent
{\bf Parameters}\\
\vspace{-0.1in}
\begin{description}
\item
\begin{tabular}{p{20mm}p{135mm}}
\texttt{usr\_address} & The Bluetooth address of the Mindstorm NXT.
\end{tabular}
\end{description}

\noindent
{\bf Description}\\
This function is used to connect to a Mindstorms NXT via Bluetooth. The Bluetooth address is gotten from the configuration file.\\

\noindent
{\bf Example}
\begin{verbatim}
#include <nxt.h> 
#include <stdio.h>

ChNXT nxt;
if (nxt.connectWithAddress("00:16:53:12:e7:80")){
    printf("Connection to NXT has failed.");
    exit(0);
}
    
nxt.disconnect();
\end{verbatim}

\noindent
{\bf See Also}\\
\texttt{connect()}, \texttt{disconnect()}\\
%%%%% END OF ChNXT::connectWithAddress %%%%%
