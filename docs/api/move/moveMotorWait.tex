\noindent
\vspace{5pt}
\rule{4.5in}{0.015in}\\
\noindent
{\LARGE \texttt{ChNXT::moveMotorWait()}\index{ChNXT::moveMotorWait()}}\\
%\phantomsection
\addcontentsline{toc}{subsection}{moveMotorWait()}

\noindent
{\bf Synopsis}
\begin{lstlisting}
#include <nxt.h>
int ChNXT::moveMotorWait(nxtMotorId_t id);
\end{lstlisting}

\noindent
{\bf Purpose}\\
Wait for a motor to stop moving.\\

\noindent
{\bf Return Value}\\
The function returns 0 on success and non-zero otherwise.\\

\noindent
{\bf Parameters}
\vspace{-0.1in}
\begin{description}
\item               
\begin{tabular}{p{10 mm}p{145 mm}}
\texttt{id} & The motor number to wait for. \\
\end{tabular}
\end{description}

\noindent
{\bf Description}\\
This function is used to wait for a motor motion to finish. Functions such as
\texttt{moveNB()} and \texttt{moveMotorNB()} do not wait for a motor to finish
moving before continuing to allow multiple motors to move at the same time. The
\texttt{moveMotorWait()} function is used to wait for a 
robotic motor motion to complete.

Please note that if this function is called after a motor has been commanded to
turn indefinitely, this function may never return and your program may hang.\\

\noindent
{\bf Example}
%Please see the example in Section \ref{sec:democode} on page \pageref{sec:democode}.\\
%\noindent

\noindent
{\bf See Also}\\
\texttt{moveWait()}

%\CPlot::\DataThreeD(), \CPlot::\DataFile(), \CPlot::\Plotting(), \plotxy().\\
