\noindent
\vspace{5pt}
\rule{4.5in}{0.015in}\\
\noindent
{\LARGE \texttt{ChNXT::moveJoint()} \index{ChNXT::moveJoint()}}\\
{\LARGE \texttt{ChNXT::moveJointNB()} \index{ChNXT::moveJointNB()}}\\

\addcontentsline{toc}{subsection}{moveJoint()}
\addcontentsline{toc}{subsection}{moveJointNB()}

\noindent
{\bf Synopsis}
\vspace{-8pt}
\begin{verbatim}
#include <nxt.h>
int ChNXT::moveJoint(nxtJointId_t id, double angle);
int ChNXT::moveJointNB(nxtJointId_t id, double angle);
\end{verbatim}

\noindent
{\bf Purpose}\\
Move a joint on the robot by a specified angle with respect to the
current position.\\

\noindent
{\bf Return Value}\\
The function returns 0 on success and non-zero otherwise.\\

\noindent
{\bf Parameters}\\
\vspace{-0.1in}
\begin{description}
\item               
\begin{tabular}{p{20 mm}p{135 mm}}
\texttt{id} & The joint number to move. \\
\texttt{angle} & The desired angle the joint need to move. \\
\end{tabular}
\end{description}

\noindent
{\bf Description}\\
\vspace{-12pt}
\begin{quote}
{\bf ChNXT::moveJoint()}\\
This function commands the motor to move by an angle relative to 
the joint's current position at the joints current speed setting.
The current motor speed ratio may be set with the 
\texttt{setJointSpeedRatio()} member function. Please note that if
the motor speed is set to zero, the motor will not move after 
calling the \texttt{moveJoint()} function. 

{\bf ChNXT::moveJointNB()}\\
This function commands the motor to move by an angle relative to 
the joint's current position at the joints current speed setting.
The current motor speed ratio may be set with the
\texttt{setJointSpeedRatio()} member function. Please note that if
the motor speed is set to zero, the motor will not move after 
calling the \texttt{moveJointNB()} function. 

The function \texttt{moveJointNB()} is the non-blocking version of
the \texttt{moveJoint()} function, which means that the function 
will return immediately and the physical robot motion will occur 
asynchronously. For more details on blocking and non-blocking 
functions, please refer to Section \ref{sec:block_nonblock} on page 
\pageref{sec:block_nonblock}.\\
\end{quote}

\noindent
{\bf Example}
\begin{verbatim}
#include <nxt.h> 
#include <stdio.h>

ChNXT nxt;

if (nxt.connectWithAddress("00:16:53:12:e7:80")){
    printf("Connection to NXT has failed.");
    exit(0);
}
 
/* setup speed for joint1 */
nxt.setJointSpeedRatio(ROBOT_JOINT1, 40);

/* move joint1 360 degrees */
nxt.moveJointNB(ROBOT_JOINT1, 360);
nxt.moveJointWait();

/* move joint1 360 degrees */
nxt.moveJoint(ROBOT_JOINT1, 360);

nxt.disconnect();
\end{verbatim}

%\noindent
%{\bf Output}\\
%None.\\
%\\
\noindent
{\bf See Also}\\
%%%%% END OF ChNXT::moveJoint %%%%%
