\noindent
\vspace{5pt}
\rule{4.5in}{0.015in}\\
\noindent
{\LARGE \texttt{ChNXT::moveToZero()} \index{ChNXT::moveToZero()}}\\
{\LARGE \texttt{ChNXT::moveToZeroNB()} \index{ChNXT::moveToZeroNB()}}\\

\addcontentsline{toc}{subsection}{moveToZero()}
\addcontentsline{toc}{subsection}{moveToZeroNB()}

\noindent
{\bf Synopsis}
\begin{lstlisting}
#include <nxt.h>
int ChNXT::moveToZero();
int ChNXT::moveToZeroNB();
\end{lstlisting}

\noindent
{\bf Purpose}\\
Move all joints of Lego Mindstorms NXT to their absolute zero position. \\

\noindent
{\bf Return Value}\\
The function returns 0 on success and non-zero otherwise.\\

\noindent
{\bf Parameters}\\
None.\\

\noindent
{\bf Description}\\
\vspace{-12pt}
\begin{quote}
{\bf ChNXT::moveToZero()}\\
This function moves all of the joints of a NXT to their zero 
position.

{\bf ChNXT::moveToZeroNB()}\\
This function moves all of the joints of a NXT to their zero 
position.

The function \texttt{moveToZeroNB()} is the non-blocking version 
of the \texttt{moveToZero()} function, which means that the 
function will return immediately and the physical robot motion 
will occur asynchronously. For more details on blocking and 
non-blocking functions, please refer to Section \ref{sec:block_nonblock}
on page \pageref{sec:block_nonblock}.\\
\end{quote}

\noindent
{\bf Example}
\begin{lstlisting}
#include <nxt.h> 

ChNXT nxt;

if (nxt.connectWithAddress("00:16:53:12:e7:80")){
    printf("Connection to NXT has failed.");
    exit(-1);
}
 
/* setup speed for all three joints */
nxt.setJointSpeeds(60, 40, 40);

/* move to zero by the non-blocking function*/
nxt.moveToZeroNB();
nxt.moveWait();

/* move to zero by the blocking function*/
nxt.moveToZero();
\end{lstlisting}

%\noindent
%{\bf Output}\\
%None.\\
%\\
\noindent
{\bf See Also}\\
%%%%% END OF ChNXT::moveToZero %%%%%
