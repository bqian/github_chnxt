\noindent
\vspace{5pt}
\rule{4.5in}{0.015in}\\
\noindent
{\LARGE \texttt{ChNXT::moveMotorContinuousNB()} \index{ChNXT::moveMotorContinuousNB()}}\\

\addcontentsline{toc}{subsection}{moveMotorContinuousNB()}

\noindent
{\bf Synopsis}
\begin{lstlisting}
#include <nxt.h>
int ChNXT::moveMotorContinuousNB(robotMotorId_t id, int dir);
\end{lstlisting}

\noindent
{\bf Purpose}\\
Move the specific motor on the Mindstorm NXT continuously in a direction.\\

\noindent
{\bf Return Value}\\
The function returns 0 on success and non-zero otherwise.\\

\noindent
{\bf Parameters}\\
\vspace{-0.1in}
\begin{description}
\item               
\begin{tabular}{p{15 mm}p{125 mm}}
\texttt{id}      &The motor of the Mindstorm NXT.\\
\texttt{dir}     &The move direction of the motor. 1 means positive and 0 means negative direction.\\
\end{tabular}
\end{description}

\noindent
{\bf Description}\\
This function is used to rotate the specific motor of the Mindstorm NXT. The motor will move continuously until the function 
\texttt{stopOneMotor()} is called. The variable \texttt{dir} indicates the move direction of the motor. \\


\noindent
{\bf Example}
\begin{lstlisting}
#include <nxt.h> 

ChNXT nxt;

if (nxt.connectWithAddress("00:16:53:12:e7:80")){
    printf("Connection to NXT has failed.");
    exit(-1);
}
 
/* setup speed for all three motors */
nxt.setMotorSpeeds(60, 40, 40);

/* move the motor 1 on the Mindstorm NXT */
nxt.moveMotorContinuousNB(NXT_MOTORA, NXT_FORWARD);
\end{lstlisting}

%\noindent
%{\bf Output}\\
%None.\\
%\\
\noindent
{\bf See Also}\\
%%%%% END OF ChNXT::moveMotorContinuousNB %%%%%
