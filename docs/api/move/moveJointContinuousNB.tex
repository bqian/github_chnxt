\noindent
\vspace{5pt}
\rule{4.5in}{0.015in}\\
\noindent
{\LARGE \texttt{ChNXT::moveJointContinuousNB()} \index{ChNXT::moveJointContinuousNB()}}\\

\addcontentsline{toc}{subsection}{moveJointContinuousNB()}

\noindent
{\bf Synopsis}
\vspace{-8pt}
\begin{verbatim}
#include <nxt.h>
int ChNXT::moveJointContinuousNB(robotJointId_t id, int dir);
\end{verbatim}

\noindent
{\bf Purpose}\\
Move the specific joint on the Mindstorm NXT continuously in a direction.\\

\noindent
{\bf Return Value}\\
The function returns 0 on success and non-zero otherwise.\\

\noindent
{\bf Parameters}\\
\vspace{-0.1in}
\begin{description}
\item               
\begin{tabular}{p{15 mm}p{125 mm}}
\texttt{id}      &The joint of the Mindstorm NXT.\\
\texttt{dir}     &The move direction of the joint. 1 means positive and 0 means negative direction.\\
\end{tabular}
\end{description}

\noindent
{\bf Description}\\
This function is used to rotate the specific joint of the Mindstorm NXT. The joint will move continuously until the function 
\texttt{stopOneJoint()} is called. The variable \texttt{dir} indicates the move direction of the joint. \\


\noindent
{\bf Example}
\begin{verbatim}
#include <nxt.h> 
#include <stdio.h>

ChNXT nxt;

if (nxt.connectWithAddress("00:16:53:12:e7:80")){
    printf("Connection to NXT has failed.");
    exit(0);
}
 
/* setup speed for all three joints */
nxt.setJointSpeeds(60, 40, 40);

/* move the joint 1 on the Mindstorm NXT */
nxt.moveJointContinuousNB(ROBOT_JOINT1, 1);

nxt.disconnect();
\end{verbatim}

%\noindent
%{\bf Output}\\
%None.\\
%\\
\noindent
{\bf See Also}\\
%%%%% END OF ChNXT::moveJointContinuousNB %%%%%
