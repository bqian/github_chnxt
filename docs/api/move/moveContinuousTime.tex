\noindent
\vspace{5pt}
\rule{4.5in}{0.015in}\\
\noindent
{\LARGE \texttt{ChNXT::moveContinuousTime()} \index{ChNXT::moveContinuousTime()}}\\

\addcontentsline{toc}{subsection}{moveContinuousTime()}

\noindent
{\bf Synopsis}
\begin{lstlisting}
#include <nxt.h>
int ChNXT::moveContinuousTime(nxtJointState_t dirA, 
                              nxtJointState_t dirB, 
                              nxtJointState_t dirC, 
                              double seconds);
\end{lstlisting}

\noindent
{\bf Purpose}\\
Move the joints of a robot continuously in the specified directions for some amount of time.\\

\noindent
{\bf Return Value}\\
The function returns 0 on success and non-zero otherwise.\\

\noindent
{\bf Parameters}\\
Each direction parameter specifies the direction the joint should 
move. The types are enumerated in \texttt{nxt.h} and have the 
following values:\\ \\
\noindent
\begin{tabular}{p{1.75in}p{4.5in}} \hline 
Value & Description \\
\hline 
\texttt{NXT\_FORWARD} & This value indicates that the joint is currently moving forward. \\
\texttt{NXT\_BACKWARD}& This value indicates that the joint is currently moving backward.\\
\hline
\end{tabular}\\

 
\noindent
The \texttt{seconds} parameter is the time to perform the movement, in seconds.
\\

\noindent
{\bf Description}\\
This function causes joints of a robot to begin moving. The joints
will continue moving until the joint hits a joint limit, or the 
time specified in the \texttt{seconds} parameter is reached. This 
function will block until the motion is completed.\\

\noindent
{\bf Example}
\begin{lstlisting}
#include <nxt.h> 

ChNXT nxt;

if (nxt.connectWithAddress("00:16:53:12:e7:80")){
    printf("Connection to NXT has failed.");
    exit(0);
}
 
/* setup speed for all three joints */
nxt.setJointSpeedRatios(0.4, 0.4, 0.4);

/* move all joints on the Mindstorm NXT in 10 seconds*/
nxt.moveContinuousTime(NXT_FORWARD, NXT_FORWARD, 
                       NXT_FORWARD, 10);
\end{lstlisting}

%\noindent
%{\bf Output}\\
%None.\\
%\\
\noindent
{\bf See Also}\\
%%%%% END OF ChNXT::moveContinuousTime %%%%%
