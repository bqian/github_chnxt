\noindent
\vspace{5pt}
\rule{4.5in}{0.015in}\\
\noindent
{\LARGE \texttt{ChNXT::moveWait()}\index{ChNXT::moveWait()}}\\
%\phantomsection
\addcontentsline{toc}{subsection}{moveWait()}

\noindent
{\bf Synopsis}
\begin{lstlisting}
#include <nxt.h>
int ChNXT::moveWait();
\end{lstlisting}

\noindent
{\bf Purpose}\\
Wait for all motors to stop moving.\\

\noindent
{\bf Return Value}\\
The function returns 0 on success and non-zero otherwise.\\

\noindent
{\bf Description}\\
This function is used to wait for all motor motions to finish. Functions such as
\texttt{move()} and \texttt{moveTo()} do not wait for a motor to finish
moving before continuing to allow multiple motors to move at the same time. The
\texttt{moveWait()} function is used to wait for robotic motions to complete.\\

\noindent
Please note that if this function is called after a motor has been commanded to
turn indefinitely, this function may never return and your program may hang.\\

\noindent
{\bf Example}
\begin{lstlisting}
#include <nxt.h> 

ChNXT nxt;

if (!nxt.connectWithAddress("00:16:53:12:e7:80")){
    printf("Connection to NXT has failed.");
    exit(-1);
}
 
/* setup speed for all three motors */
nxt.setMotorSpeeds(60, 40, 40);

/* move by the non-blocking function*/
nxt.moveNB(360, 360, 360);

/* wait until motors stop moving */
nxt.moveWait();
\end{lstlisting}

%See the sample program in Section \ref{sec:democode} on page \pageref{sec:democode}.

\noindent
{\bf See Also}\\
\texttt{moveMotorWait()}

%\CPlot::\DataThreeD(), \CPlot::\DataFile(), \CPlot::\Plotting(), \plotxy().\\
