\noindent
\vspace{5pt}
\rule{4.5in}{0.015in}\\
\noindent
{\LARGE \texttt{ChNXT::moveTo()} \index{ChNXT::moveTo()}}\\
{\LARGE \texttt{ChNXT::moveToNB()} \index{ChNXT::moveToNB()}}\\

\addcontentsline{toc}{subsection}{moveTo()}
\addcontentsline{toc}{subsection}{moveToNB()}

\noindent
{\bf Synopsis}
\begin{lstlisting}
#include <nxt.h>
int ChNXT::moveTo(double angleA, double angleB, double angleC);
int ChNXT::moveToNB(double angleA, double angleB, double angleC);
\end{lstlisting}

\noindent
{\bf Purpose}\\
Move all joints of Lego Mindstorms NXT to the specified position.\\

\noindent
{\bf Return Value}\\
The function returns 0 on success and non-zero otherwise.\\

\noindent
{\bf Parameters}\\
\vspace{-0.1in}
\begin{description}
\item               
\begin{tabular}{p{15 mm}p{105 mm}}
\texttt{angle1} & The absolute position to move joint A, expressed in degrees. \\
\texttt{angle2} & The absolute position to move joint B, expressed in degrees. \\
\texttt{angle3} & The absolute position to move joint C, expressed in degrees. \\
\end{tabular}
\end{description}

\noindent
{\bf Description}\\
\vspace{-12pt}
\begin{quote}
{\bf ChNXT::moveTo()}\\
This function moves all of the joints of a robot to the specified 
absolute positions. 

{\bf ChNXT::moveToNB()}\\
This function moves all of the joints of a robot to the specified 
absolute positions. 

The function \texttt{moveToNB()} is the non-blocking version of
the \texttt{moveTo()} function, which means that the function will
return immediately and the physical robot motion will occur 
asynchronously. For more details on blocking and non-blocking 
functions, please refer to Section \ref{sec:block_nonblock} on page 
\pageref{sec:block_nonblock}.\\
\end{quote}

\noindent
{\bf Example}
\begin{lstlisting}
#include <nxt.h> 

ChNXT nxt;

if (nxt.connectWithAddress("00:16:53:12:e7:80")){
    printf("Connection to NXT has failed.");
    exit(-1);
}
 
/* setup speed for all three joints */
nxt.setJointSpeeds(40, 40, 40);

/* move by the non-blocking function*/
nxt.moveToNB(360, 360, 360);
nxt.moveWait();

/* move by the blocking function*/
nxt.moveTo(360, 360, 360);
\end{lstlisting}

\noindent
{\bf See Also}\\
{\tt move(), moveNB()}\\
%%%%% END OF ChNXT::move %%%%%
