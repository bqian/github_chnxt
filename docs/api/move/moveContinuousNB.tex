\noindent
\vspace{5pt}
\rule{4.5in}{0.015in}\\
\noindent
{\LARGE \texttt{ChNXT::moveContinuousNB()} \index{ChNXT::moveContinuousNB()}}\\

\addcontentsline{toc}{subsection}{moveContinuousNB()}

\noindent
{\bf Synopsis}
\begin{lstlisting}
#include <nxt.h>
int ChNXT::moveContinuousNB(nxtJointState_t dirA, 
                            nxtJointState_t dirB, 
                            nxtJointState_t dirC); 
\end{lstlisting}

\noindent
{\bf Purpose}\\
Move the joints of a robot continuously in the specified directions.\\

\noindent
{\bf Return Value}\\
The function returns 0 on success and non-zero otherwise.\\

\noindent
{\bf Parameters}\\
Each parameter specifies the direction the joint should move. The types
are enumerated in \texttt{mobot.h} and have the following values:\\
\\
\noindent
\begin{tabular}{p{1.75in}p{4.5in}} \hline 
Value & Description \\
\hline 
\texttt{ROBOT\_FORWARD} & This value indicates that the joint is currently moving forward. \\
\texttt{ROBOT\_BACKWARD}& This value indicates that the joint is currently moving backward.\\
\hline
\end{tabular}\\


\noindent
More documentation about these types may be found at Section
\ref{sec:nxtJointState_t} on page
\pageref{sec:nxtJointState_t}.

\noindent
{\bf Description}\\
This function causes joints of a robot to begin moving at the 
previously set speed. The joints will continue moving until the 
joint hits a joint limit, or the joint is stopped by setting the 
speed to zero. This function is a non-blocking function.\\

\noindent
{\bf Example}
\begin{lstlisting}
#include <nxt.h> 

ChNXT nxt;

if (nxt.connectWithAddress("00:16:53:12:e7:80")){
    printf("Connection to NXT has failed.");
    exit(0);
}
 
/* setup speed for all three joints */
nxt.setJointSpeeds(60, 40, 40);

/* move all joints on the Mindstorm NXT */
nxt.moveContinuousNB(NXT_FORWARD, NXT_FORWARD, NXT_FORWARD);
\end{lstlisting}

%\noindent
%{\bf Output}\\
%None.\\
%\\
\noindent
{\bf See Also}\\
%%%%% END OF ChNXT::moveContinuousNB %%%%%
