\noindent
\vspace{5pt}
\rule{4.5in}{0.015in}\\
\noindent
{\LARGE \texttt{ChNXT::getMotorState()}\index{ChNXT::getMotorState()}}\\
%\phantomsection
\addcontentsline{toc}{subsection}{getMotorState()}

\noindent
{\bf Synopsis}
\begin{lstlisting}
#include <nxt.h>
int ChNXT::getMotorState(nxtMotorId_t id, nxtMotorState_t &state);
\end{lstlisting}

\noindent
{\bf Purpose}\\
Determine whether a motor is moving or not.\\

\noindent
{\bf Return Value}\\
The function returns 0 on success and non-zero otherwise.\\

\noindent
{\bf Parameters}
\vspace{-0.1in}
\begin{description}
\item               
\begin{tabular}{p{10 mm}p{145 mm}}
\texttt{id} & The motor number. This is an enumerated type 
discussed in Section \ref{sec:nxtMotorId_t} on Page~\pageref{sec:nxtMotorId_t}.\\
\texttt{state} & An integer variable which will be overwritten with the current 
state of the motor. This is an enumerated type discussed in 
Section~\ref{sec:nxtMotorState_t} on Page~\pageref{sec:nxtMotorState_t}.\\
\end{tabular}
\end{description}

\noindent
{\bf Description}\\
This function is used to determine the current state of a motor. Valid states are listed below.\\
\\
\input{motorStateTable}

\noindent
{\bf Example}
\begin{lstlisting}
#include <nxt.h>

ChNXT nxt;
nxtMotorState_t status;

nxt.connect();

nxt.getMotorState(NXT_MOTORA, status);
if(status == 0)
    printf("MotorA is not moving.\n");
else
    printf("MotorA is moving.\n");
\end{lstlisting}
%\noindent

\noindent
{\bf See Also}\\
\texttt{isMoving()}\\
%\CPlot::\DataThreeD(), \CPlot::\DataFile(), \CPlot::\Plotting(), \plotxy().\\
