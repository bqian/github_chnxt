\noindent
\vspace{5pt}
\rule{4.5in}{0.015in}\\
\noindent
{\LARGE \texttt{ChNXT::getJointAngle()}\index{ChNXT::getJointAngle()}}\\
%\phantomsection
\addcontentsline{toc}{subsection}{getJointAngle()}

\noindent
{\bf Synopsis}
\vspace{-8pt}
\begin{verbatim}
#include <nxt.h>
int ChNXT::getJointAngle(nxtJointId_t id, double &angle);
\end{verbatim}

\noindent
{\bf Purpose}\\
Retrieve a robot joint's current angle.\\

\noindent
{\bf Return Value}\\
The function returns 0 on success and non-zero otherwise.\\

\noindent
{\bf Parameters}\\
\vspace{-0.1in}
\begin{description}
\item               
\begin{tabular}{p{15 mm}p{145 mm}}
\texttt{id} & The joint number. This is an enumerated type 
discussed in Section \ref{sec:nxtJointId_t} on page
\pageref{sec:nxtJointId_t}.\\
\texttt{angle} & A variable to store the current angle of the robot
motor. The contents of this variable will be overwritten with a value that
represents the motor's angle in degrees.  \\
\end{tabular}
\end{description}

\noindent
{\bf Description}\\
This function gets the current motor angle of a NXT's motor. The
angle returned is in units of degrees and is accurate to roughly $\pm1$
degrees. The function \texttt{getJointAngle()} always returns an angle 
from 0 to $\infty$ degrees.\\

\noindent
{\bf Example}\\
\begin{verbatim}
#include <nxt.h>

ChNXT nxt;
double angle;

nxt.connect();

nxt.getJointAngle(ROBOT_JOINT1, angle);

printf("The angle of joint1 is: %lf\n", angle);
\end{verbatim}

\noindent
{\bf See Also}\\

%\CPlot::\DataThreeD(), \CPlot::\DataFile(), \CPlot::\Plotting(), \plotxy().\\
