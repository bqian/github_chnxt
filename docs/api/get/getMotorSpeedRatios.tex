\noindent
\vspace{5pt}
\rule{4.5in}{0.015in}\\
\noindent
{\LARGE \texttt{ChNXT::getMotorSpeedRatios()}\index{ChNXT::getMotorSpeedRatios()}}\\
%\phantomsection
\addcontentsline{toc}{subsection}{getMotorSpeedRatios()}

\noindent
{\bf Synopsis}
\begin{lstlisting}
#include <nxt.h>
int ChNXT::getMotorSpeedRatios(double &ratio1, double &ratio2, double &ratio3);
\end{lstlisting}

\noindent
{\bf Purpose}\\
Get the speed ratio settings of all motors on the Lego Mindstorms NXT.\\

\noindent
{\bf Return Value}\\
The function returns 0 on success and non-zero otherwise.\\

\noindent
{\bf Parameters}
\vspace{-0.1in}
\begin{description}
\item               
\begin{tabular}{p{10 mm}p{145 mm}}
\texttt{ratio1} & The address of a variable to store the speed ratio of motor A.\\
\texttt{ratio2} & The address of a variable to store the speed ratio of motor B.\\
\texttt{ratio3} & The address of a variable to store the speed ratio of motor C.\\
\end{tabular}
\end{description}

\noindent
{\bf Description}\\
This function is used to retrieve all four motor speed ratio settings of a Lego
Mindstorms NXT simultaneously. The speed ratios are as a value from 0 to 1. \\

\noindent
{\bf Example}
\begin{lstlisting}
#include <nxt.h>

ChNXT nxt;
double ratio1, ratio2, ratio3;

nxt.connect();

nxt.getMotorSpeedRatio(ratio1, ratio2, ratio3);

printf("The speed ratio of motorA is: %lf\n", ratio1);
printf("The speed ratio of motorB is: %lf\n", ratio2);
printf("The speed ratio of motorC is: %lf\n", ratio3);
\end{lstlisting}

\noindent
{\bf See Also}\\
\texttt{getMotorSpeedRatio(), setMotorSpeedRatio()}

%\CPlot::\DataThreeD(), \CPlot::\DataFile(), \CPlot::\Plotting(), \plotxy().\\
