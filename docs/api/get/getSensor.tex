\noindent
\vspace{5pt}
\rule{4.5in}{0.015in}\\
\noindent
{\LARGE \texttt{ChNXT::getSensor()}\index{ChNXT::getSensor()}}\\
%\phantomsection
\addcontentsline{toc}{subsection}{getSensor()}

\noindent
{\bf Synopsis}
\begin{lstlisting}
#include <nxt.h>
int ChNXT::getSensor(nxtSensorId_t id, double &value);
\end{lstlisting}

\noindent
{\bf Purpose}\\
Retrieve a NXT sensor's current value.\\

\noindent
{\bf Return Value}\\
The function returns 0 on success and non-zero otherwise.\\

\noindent
{\bf Parameters}\\
\vspace{-0.1in}
\begin{description}
\item               
\begin{tabular}{p{15 mm}p{145 mm}}
    {\tt id} & The sensor port. This is an enumerated type discussed in 
    Section~\ref{sec:nxtSensorPort_t} on Page~\pageref{sec:nxtSensorPort_t}.\\
    {\tt value} & A variable to store the current value of the NXT sensor. The 
    contents of this variable will be overwritten with the sensor's value.  \\
\end{tabular}
\end{description}

\noindent
{\bf Description}\\
This function gets the current sensor's value of a NXT.\\

\noindent
{\bf Example}
\begin{lstlisting}
#include <nxt.h>

ChNXT nxt;
int value;

nxt.connect();

nxt.setSensor(NXT_SENSORPORT4, NXT_SENSORTYPE_ULTRASONIC,
		NXT_SENSORMODE_RAWMODE);

nxt.getSensor(NXT_SENSORPORT4, value);

printf("The value of sensor 4 is: %d\n", value);
\end{lstlisting}

\noindent
{\bf See Also}\\

%\CPlot::\DataThreeD(), \CPlot::\DataFile(), \CPlot::\Plotting(), \plotxy().\\
