\noindent
\vspace{5pt}
\rule{4.5in}{0.015in}\\
\noindent
{\LARGE \texttt{ChNXT::getSensor()}\index{ChNXT::getSensor()}}\\
%\phantomsection
\addcontentsline{toc}{subsection}{getSensor()}

\noindent
{\bf Synopsis}
\vspace{-8pt}
\begin{verbatim}
#include <nxt.h>
int ChNXT::getSensor(nxtSensorId_id, double &value);
\end{verbatim}

\noindent
{\bf Purpose}\\
Retrieve a NXT sensor's current value.\\

\noindent
{\bf Return Value}\\
The function returns 0 on success and non-zero otherwise.\\

\noindent
{\bf Parameters}\\
\vspace{-0.1in}
\begin{description}
\item               
\begin{tabular}{p{15 mm}p{145 mm}}
\texttt{id} & The sensor port. This is an enumerated type 
discussed in Section \ref{sec:nxtSensorId_t} on page
\pageref{sec:nxtSensorId_t}.\\
\texttt{value} & A variable to store the current value of the NXT 
sensor. The contents of this variable will be overwritten with the 
sensor's value.  \\
\end{tabular}
\end{description}

\noindent
{\bf Description}\\
This function gets the current sensor's value of a NXT.\\

\noindent
{\bf Example}\\
\begin{verbatim}
#include <nxt.h>
#include <stdio.h>

ChNXT nxt;
int value;

nxt.connect();

nxt.setSensor(NXT_SENSORPORT4, NXT_SENSORTYPE_ULTRASONIC,
		NXT_SENSORMODE_RAWMODE);

nxt.getSensor(NXT_SENSORPORT4, value);

printf("The value of sensor 4 is: %d\n", value);

nxt.disconnect();
\end{verbatim}

\noindent
{\bf See Also}\\

%\CPlot::\DataThreeD(), \CPlot::\DataFile(), \CPlot::\Plotting(), \plotxy().\\
