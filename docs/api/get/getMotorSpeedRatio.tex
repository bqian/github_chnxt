\noindent
\vspace{5pt}
\rule{4.5in}{0.015in}\\
\noindent
{\LARGE \texttt{ChNXT::getMotorSpeedRatio()}\index{ChNXT::getMotorSpeedRatio()}}\\
%\phantomsection
\addcontentsline{toc}{subsection}{getMotorSpeedRatio()}

\noindent
{\bf Synopsis}
\begin{lstlisting}
#include <nxt.h>
int ChNXT::getMotorSpeedRatio(nxtMotorId_t id, double &ratio);
\end{lstlisting}

\noindent
{\bf Purpose}\\
Get the speed ratio settings of a motor on the Lego Mindstorms NXT.\\

\noindent
{\bf Return Value}\\
The function returns 0 on success and non-zero otherwise.\\

\noindent
{\bf Parameters}
\vspace{-0.1in}
\begin{description}
\item               
\begin{tabular}{p{10 mm}p{145 mm}}
\texttt{id} & Retrieve the speed ratio setting of this motor. This is an 
enumerated type discussed in Section \ref{sec:nxtMotorId_t} on page
\pageref{sec:nxtMotorId_t}.\\
\texttt{ratio} & A variable of type double. The value of this variable will
be overwritten with the current speed ratio setting of the motor.
\end{tabular}
\end{description}

\noindent
{\bf Description}\\
This function is used to find the speed ratio setting of a motor. The speed
ratio setting of a motor is the percentage of the maximum motor speed, and the
value ranges from 0 to 1. In other words, if the ratio is set to 0.5, the motor 
will turn at 50\% of its maximum angular velocity while moving continuously
or moving to a new goal position.\\

\noindent
{\bf Example}
\begin{lstlisting}
#include <nxt.h>

ChNXT nxt;
double ratio;

nxt.connect();

nxt.getMotorSpeedRatio(NXT_MOTORA, ratio);

printf("The speed ratio of motorA is: %lf\n", ratio);
\end{lstlisting}

\noindent
{\bf See Also}\\
\texttt{setMotorSpeedRatio(), getMotorSpeedRatio()}

%\CPlot::\DataThreeD(), \CPlot::\DataFile(), \CPlot::\Plotting(), \plotxy().\\
