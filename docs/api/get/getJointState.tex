\noindent
\vspace{5pt}
\rule{4.5in}{0.015in}\\
\noindent
{\LARGE \texttt{ChNXT::getJointState()}\index{ChNXT::getJointState()}}\\
%\phantomsection
\addcontentsline{toc}{subsection}{getJointState()}

\noindent
{\bf Synopsis}
\vspace{-8pt}
\begin{verbatim}
#include <nxt.h>
int ChNXT::getJointState(nxtJointId_t id, nxtJointState_t &state);
\end{verbatim}

\noindent
{\bf Purpose}\\
Determine whether a motor is moving or not.\\

\noindent
{\bf Return Value}\\
The function returns 0 on success and non-zero otherwise.\\

\noindent
{\bf Parameters}
\vspace{-0.1in}
\begin{description}
\item               
\begin{tabular}{p{10 mm}p{145 mm}}
\texttt{id} & The joint number. This is an enumerated type 
discussed in Section \ref{sec:nxtJointId_t} on page
\pageref{sec:nxtJointId_t}.\\
\texttt{state} & An integer variable which will be overwritten with the current state of the motor. 
This is an enumerated type 
discussed in Section \ref{sec:nxtJointState_t} on page
\pageref{sec:nxtJointState_t}.
\end{tabular}
\end{description}

\noindent
{\bf Description}\\
This function is used to determine the current state of a motor. Valid states are listed below.\\
\\
\\
\noindent
\begin{tabular}{p{1.75in}p{4.5in}} \hline 
Value & Description \\
\hline 
\texttt{NXT\_FORWARD} & This value indicates that the joint is currently moving forward. \\
\texttt{NXT\_BACKWARD}& This value indicates that the joint is currently moving backward.\\
\hline
\end{tabular}\\



\noindent
{\bf Example}\\
\begin{verbatim}
#include <nxt.h>

ChNXT nxt;
nxtJointState_t status;

nxt.connect();

nxt.getJointState(ROBOT_JOINT1, status);
if(status == 0)
    printf("Joint1 is not moving.\n");
else
    printf("Joint1 is moving.\n");
\end{verbatim}
%\noindent

\noindent
{\bf See Also}\\
\texttt{isMoving()}\\
%\CPlot::\DataThreeD(), \CPlot::\DataFile(), \CPlot::\Plotting(), \plotxy().\\
