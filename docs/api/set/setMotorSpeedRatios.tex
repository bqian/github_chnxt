\noindent
\vspace{5pt}
\rule{4.5in}{0.015in}\\
\noindent
{\LARGE \texttt{ChNXT::setMotorSpeedRatios()} \index{ChNXT::setMotorSpeedRatios()}}\\

\addcontentsline{toc}{subsection}{setMotorSpeedRatios()}

\noindent
{\bf Synopsis}
\begin{lstlisting}
#include <nxt.h>
int ChNXT::setMotorSpeedRatios(double ratioA, double ratioB, double ratioC);
\end{lstlisting}

\noindent
{\bf Purpose}\\
Set the speed ratio setting of all motors on the Lego Mindstorms NXT.\\

\noindent
{\bf Return Value}\\
The function returns 0 on success and non-zero otherwise.\\

\noindent
{\bf Parameters}\\
\vspace{-0.1in}
\begin{description}
\item
\begin{tabular}{ p{20mm}p{135mm} }
\texttt{ratioA}&The speed ratio setting for the first motor.\\
\texttt{ratioB}&The speed ratio setting for the second motor.\\
\texttt{ratioC}&The speed ratio setting for the third motor.\\
\end{tabular}
\end{description}

\noindent
{\bf Description}\\
This function is used to simultaneously set the speed ratio settings 
of all motors on the Lego Mindstorms NXT. The speed ratio setting of
a motor is the percentage of the maximum motor speed, and the value 
ranges from 0 to 1.\\

\noindent
{\bf Example}
\begin{lstlisting}
#include <nxt.h> 

ChNXT nxt;

if (nxt.connectWithAddress("00:16:53:12:e7:80")){
    printf("Connection to NXT has failed.");
    exit(-1);
}
    
/* set the speed ratio settings for all motors*/
nxt.setMotorSpeedRatios(0.5, 0.4, 0.4);
\end{lstlisting}

\noindent
{\bf See Also}\\
\texttt{setMotorSpeedRatio()}\\
%%%%% END OF ChNXT::connect %%%%%
