\noindent
\vspace{5pt}
\rule{4.5in}{0.015in}\\
\noindent
{\LARGE \texttt{ChNXT::setJointSpeedRatio()} \index{ChNXT::setJointSpeedRatio()}}\\

\addcontentsline{toc}{subsection}{setJointSpeedRatio()}

\noindent
{\bf Synopsis}
\vspace{-8pt}
\begin{verbatim}
#include <nxt.h>
int ChNXT::setJointSpeedRatio(nxtJointId_t id, double ratio);
\end{verbatim}

\noindent
{\bf Purpose}\\
Set the speed ratio setting of a joint on the Lego Mindstorms NXT.\\

\noindent
{\bf Return Value}\\
The function returns 0 on success and non-zero otherwise.\\

\noindent
{\bf Parameters}\\
\vspace{-0.1in}
\begin{description}
\item
\begin{tabular}{ p{20mm}p{135mm} }
\texttt{id}&The port of the sensor locate on the Mindstorm NXT.\\
\texttt{ratio}&A variable of type double with a value from 0 to 1.\\
\end{tabular}
\end{description}

\noindent
{\bf Description}\\
This function is used to set the speed ratio setting of a joint on
the Lego Mindstorms NXT. The speed ratio setting of a joint is the
percentage of the maximum joint speed, and the value ranges from 
0 to 1. In other words, if the ratio is set to 0.5, the joint will
turn at 50\% of its maximum angular velocity while moving 
continuously or moving to a new goal position.\\

\noindent
{\bf Example}
\begin{verbatim}
#include <nxt.h> 

ChNXT nxt;

if (nxt.connectWithAddress("00:16:53:12:e7:80")){
    printf("Connection to NXT has failed.");
    exit(-1);
}
    
/* set the speed ratio setting for joint1 on port0 */
nxt.setJointSpeedRatio(ROBOT_JOINT1, 0.5);
\end{verbatim}
\noindent
{\bf See Also}\\
\texttt{setJointSpeedRatios()}\\
%%%%% END OF ChNXT::connect %%%%%
