\noindent
\vspace{5pt}
\rule{4.5in}{0.015in}\\
\noindent
{\LARGE \texttt{ChNXT::setJointToZero()} \index{ChNXT::setJointToZero()}}\\

\addcontentsline{toc}{subsection}{setJointToZero()}

\noindent
{\bf Synopsis}
\begin{lstlisting}
#include <nxt.h>
int ChNXT::setJointToZero(nxtJointId_t id);
\end{lstlisting}

\noindent
{\bf Purpose}\\
Reset the tachometer count for a single joint on Mindstorms NXT.\\

\noindent
{\bf Return Value}\\
The function returns 0 on success and non-zero otherwise.\\

\noindent
{\bf Parameters}\\
\vspace{-0.1in}
\begin{description}
\item
\begin{tabular}{p{20mm}p{135mm}}
\texttt{id} &The port of the joint locate on the Mindstorm NXT.\\
\end{tabular}
\end{description}

\noindent
{\bf Description}\\
This function is used to reset the tachometer count for a joint on the Mindstorms NXT.\\

\noindent
{\bf Example}
\begin{lstlisting}
#include <nxt.h> 

ChNXT nxt;

if (nxt.connectWithAddress("00:16:53:12:e7:80")){
    printf("Connection to NXT has failed.");
    exit(-1);
}
    
nxt.setJointToZero(NXT_JOINTA);
\end{lstlisting}

%\noindent
%{\bf Output}\\
%None.\\
%\\
\noindent
{\bf See Also}\\
\texttt{setJointToZeros()}\\
%%%%% END OF ChNXT::setJointRelativeZero %%%%%
