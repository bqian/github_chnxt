\noindent
\vspace{5pt}
\rule{4.5in}{0.015in}\\
\noindent
{\LARGE \texttt{ChNXT::setJointRelativeZero()} \index{ChNXT::setJointRelativeZero()}}\\

\addcontentsline{toc}{subsection}{setJointRelativeZero()}

\noindent
{\bf Synopsis}
\vspace{-8pt}
\begin{verbatim}
#include <nxt.h>
int ChNXT::setJointRelativeZero(robotJointId_t id);
\end{verbatim}

\noindent
{\bf Purpose}\\
Reset the relative tachometer count for a single joint on Mindstorms NXT.\\

\noindent
{\bf Return Value}\\
The function returns 0 on success and non-zero otherwise.\\

\noindent
{\bf Parameters}\\
\vspace{-0.1in}
\begin{description}
\item
\begin{tabular}{p{20mm}p{135mm}}
\texttt{id} &The port of a joint located on the Mindstorm NXT.\\
\end{tabular}
\end{description}

\noindent
{\bf Description}\\
This function is used to reset the relative tachometer count for a joint on the Mindstorms NXT.\\

\noindent
{\bf Example}
\begin{verbatim}
#include <nxt.h> 

ChNXT nxt;

if (!nxt.connectWithAddress("00:16:53:12:e7:80")){
    printf("Connection to NXT has failed.");
    exit(-1);
}
    
nxt.setJointRelativeZero(ROBOT_JOINT1);
\end{verbatim}

\noindent
{\bf Output}\\
None.\\
\\
%%%%% END OF ChNXT::setJointRelativeZero %%%%%
