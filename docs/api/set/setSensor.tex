\noindent
\vspace{5pt}
\rule{4.5in}{0.015in}\\
\noindent
{\LARGE \texttt{ChNXT::setSensor()} \index{ChNXT::setSensor()}}\\

\addcontentsline{toc}{subsection}{setSensor()}

\noindent
{\bf Synopsis}
\vspace{-8pt}
\begin{verbatim}
#include <nxt.h>
int ChNXT::setSensor(nxtSensorId_t id,
                     nxtSensorType type, 
                     nxtSensorMode mode);
\end{verbatim}

\noindent
{\bf Purpose}\\
Set up a sensor of Lego mindstorms NXT.\\

\noindent
{\bf Return Value}\\
The function returns 0 on success and non-zero otherwise.\\

\noindent
{\bf Parameters}\\
\vspace{-0.1in}
\begin{description}
\item
\begin{tabular}{ p{20mm}p{135mm} }
\texttt{id}&The port of the sensor locate on the Mindstorm NXT.\\
\texttt{type}   &The type of the sensor. \\
\texttt{mode}   &The mode of the sensor. \\
\end{tabular}
\end{description}

\noindent
{\bf Description}\\
This function is used to set up a sensor on the Mindstorm with a specific type and mode of the sensor. See the Macros section for 
the details of ports, types and modes.\\

\noindent
{\bf Example}
\begin{verbatim}
#include <nxt.h> 
#include <stdio.h>

ChNXT nxt;
int status_1=2;
int status_2=2;

if (nxt.connectWithAddress("00:16:53:12:e7:80")){
    printf("Connection to NXT has failed.");
    exit(0);
}
    
/* setup port 1 as touch sensor */
status_1=nxt.setSensor(SENSOR_PORT1, 
        SENSOR_TYPE_TOUCH, SENSOR_MODE_BOOLEANMODE);
if(status_1){
    printf("Connection to touch sensor has failed");
}

/* setup port4 as ultrasonic sensor */
status_2=nxt.setSensor(SENSOR_PORT4, 
        SENSOR_TYPE_ULTRASONIC, SENSOR_MODE_RAWMODE);
if(status_2){
    printf("Connection to ultrasonic sensor has failed");
}

nxt.disconnect();
\end{verbatim}

\noindent
{\bf See Also}\\

%%%%% END OF ChNXT::setSensor %%%%%
