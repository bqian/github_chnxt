\noindent
\vspace{5pt}
\rule{4.5in}{0.015in}\\
\noindent
{\LARGE \texttt{ChNXT::setSensor()} \index{ChNXT::setSensor()}}\\

\addcontentsline{toc}{subsection}{setSensor()}

\noindent
{\bf Synopsis}
\begin{lstlisting}
#include <nxt.h>
int ChNXT::setSensor(nxtSensorId_t id,
                     nxtSensorType type, 
                     nxtSensorMode mode);
\end{lstlisting}

\noindent
{\bf Purpose}\\
Set up a sensor of Lego mindstorms NXT.\\

\noindent
{\bf Return Value}\\
The function returns 0 on success and non-zero otherwise.\\

\noindent
{\bf Parameters}\\
\vspace{-0.1in}
\begin{description}
\item
\begin{tabular}{ p{20mm}p{135mm} }
\texttt{id}&The port of the sensor locate on the Mindstorm NXT.\\
\texttt{type}   &The type of the sensor. \\
\texttt{mode}   &The mode of the sensor. \\
\end{tabular}
\end{description}


\noindent
{\bf NXT Sensor Types}\\
\begin{longtable}{p{5.5cm}p{10cm}}
%\begin{tabular}{p{5.5cm}p{10cm}} \hline
    \hline
Value &       Description\\
\hline
{\tt NXT\_SENSORTYPE\_SWITCH}        &Set to a switch type sensor. Touch sensor is a switch type sensor.\\
%{\tt NXT\_SENSORTYPE\_TEMPERATURE}   &Set to Temperature Sensor.\\
{\tt NXT\_SENSORTYPE\_LIGHT\_ACTIVE} &Set to Light Sensor in light active mode(LED on).\\
{\tt NXT\_SENSORTYPE\_LIGHT\_INACTIVE}&Set to Light Sensor in light inactive mode(LED off).\\
{\tt NXT\_SENSORTYPE\_SOUND\_DB}     &Set to Sound Sensor in dB.\\
{\tt NXT\_SENSORTYPE\_SOUND\_DBA}    &Set to Sound Sensor in dB with adjusted.\\
{\tt NXT\_SENSORTYPE\_LOWSPEED}      &Set to ISP type sensor.\\
{\tt NXT\_SENSORTYPE\_LOWSPEED\_9V}  &Set to ISP type sensor with 9 Voltage. The ultrasonic sensor belongs to this type of sensor.\\
%{\tt NXT\_SENSORTYPE\_HIGHSPEED}     &Not avialable now.\\
{\tt NXT\_SENSORTYPE\_COLORFULL}     &Set to Color Sensor in color detector mode.\\
{\tt NXT\_SENSORTYPE\_COLORRED}      &Set to Color Sensor in lightsensor mode with red light on.\\
{\tt NXT\_SENSORTYPE\_COLORGREEN}    &Set to Color Sensor in lightsensor mode with green light on.\\
{\tt NXT\_SENSORTYPE\_COLORBLUE}     &Set to Color Sensor in lightsensor mode with blue light on.\\
{\tt NXT\_SENSORTYPE\_COLORNONE}     &Set to Color Sensor in lightsensor mode with no light on.\\
\hline
\end{longtable}

\noindent
{\bf NXT Sensor Modes}\\
\begin{longtable}{p{6cm}p{9.5cm}}
%\begin{tabular}{p{6cm}p{9.5cm}} \hline
    \hline
Value & Description\\
\hline
{\tt NXT\_SENSORMODE\_RAWMODE}      &Get sensor value as raw mode.\\	 
{\tt NXT\_SENSORMODE\_BOOLEANMODE}  &Get sensor value as boolean mode.\\
{\tt NXT\_SENSORMODE\_TRANSITIONCNTMODE}  &Get sensor value as a number of transitions between TRUE and FALSE.\\
{\tt NXT\_SENSORMODE\_COUNTERMODE}  &Get sensor value as a number of transitions from FALSE to TRUE, then back to FALSE.\\
{\tt NXT\_SENSORMODE\_PCTFULLSCALEDMODE}  &Get sensor value as percentage of full scale reading for configured sensor type.\\
%{\tt NXT\_SENSORMODE\_CELSIUSMODE}  &Get sensor value of temperature in degrees Celsius.\\
%{\tt NXT\_SENSORMODE\_FAHRENHEITMODE}  &Get sensor value of temperature in degrees Fahrenheit.\\
\hline
%\end{tabular}
\end{longtable}

\noindent
{\bf NXT Sensor Range}\\
\begin{longtable}{p{6cm}p{9.5cm}}
    \hline
    Sensor & Range\\
\hline
Touch Sensor & 0 and 1.\\
Light Sensor & 0 to 100 (percentage).\\
Sound Sensor & 0 to 100 (percentage).\\
Ultrasonic Sensor & 0 to 80 cm.\\
     Color Sensor & 0 to 6.\\
\hline
\end{longtable}

\noindent
{\bf Description}\\
This function is used to set up a sensor on the Mindstorm with a specific type and mode of the sensor. Function {\tt getSensor()} can be used to get the sensor values for each type of sensor.\\

\noindent
{\bf Example}
\begin{lstlisting}
#include <nxt.h> 

ChNXT nxt;
int status_1=2;
int status_2=2;

if (nxt.connectWithAddress("00:16:53:12:e7:80")){
    printf("Connection to NXT has failed.");
    exit(-1);
}
    
/* setup port 1 as touch sensor */
status_1=nxt.setSensor(SENSOR_PORT1, 
        SENSOR_TYPE_TOUCH, SENSOR_MODE_BOOLEANMODE);
if(status_1){
    printf("Connection to touch sensor has failed");
}

/* setup port4 as ultrasonic sensor */
status_2=nxt.setSensor(SENSOR_PORT4, 
        SENSOR_TYPE_ULTRASONIC, SENSOR_MODE_RAWMODE);
if(status_2){
    printf("Connection to ultrasonic sensor has failed");
}
\end{lstlisting}

\noindent
{\bf See Also}\\
{\tt getSensor()}

%%%%% END OF ChNXT::setSensor %%%%%
