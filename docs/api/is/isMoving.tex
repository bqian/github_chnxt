\noindent
\vspace{5pt}
\rule{4.5in}{0.015in}\\
\noindent
{\LARGE \texttt{ChNXT::isMoving()} \index{ChNXT::isMoving()}}\\

\addcontentsline{toc}{subsection}{isMoving()}

\noindent
{\bf Synopsis}
\begin{lstlisting}
#include <nxt.h>
int ChNXT::isMoving(void);
\end{lstlisting}

\noindent
{\bf Purpose}\\
Check to see if a Lego Mindstorms NXT is currently moving any of its motors.\\

\noindent
{\bf Return Value}\\
This function returns 0 if none of the motors are being driven or 
if an error has occured, or 1 if any motor is being driven.\\

\noindent
{\bf Parameters}\\
None.\\

\noindent
{\bf Description}\\
This function is used to determine if a robot is currently moving any of its motors.\\

\noindent
{\bf Example}
\begin{lstlisting}
#include <nxt.h>

ChNXT nxt;
int moveStatus;

nxt.connect();
moveStatus = isMoving(NXT_MOTORA);

if(moveStatus)
    printf("MotorA is moving!\n");
else
    printf("MotorA is not moving!\n");
\end{lstlisting}

\noindent
{\bf See Also}\\
\texttt{getMotorState()}\\
%%%%% END OF ChNXT::connect %%%%%
