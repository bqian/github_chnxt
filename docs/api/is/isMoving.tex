\noindent
\vspace{5pt}
\rule{4.5in}{0.015in}\\
\noindent
{\LARGE \texttt{ChNXT::isMoving()} \index{ChNXT::isMoving()}}\\

\addcontentsline{toc}{subsection}{isMoving()}

\noindent
{\bf Synopsis}
\vspace{-8pt}
\begin{verbatim}
#include <nxt.h>
int ChNXT::isMoving(void);
\end{verbatim}

\noindent
{\bf Purpose}\\
Check to see if a Lego Mindstorms NXT is currently moving any of its joints.\\

\noindent
{\bf Return Value}\\
This function returns 0 if none of the joints are being driven or 
if an error has occured, or 1 if any joint is being driven.\\

\noindent
{\bf Parameters}\\
None.\\

\noindent
{\bf Description}\\
This function is used to determine if a robot is currently moving any of its joints.\\

\noindent
{\bf Example}
\begin{verbatim}
#include <nxt.h>
#include <stdio.h>

ChNXT nxt;
int moveStatus;

nxt.connect();
moveStatus = isMoving(ROBOT_JOINT1);

if(moveStatus)
    printf("Joint1 is moving!\n");
else
    printf("Joint1 is not moving!\n");

nxt.disconnect();
\end{verbatim}

\noindent
{\bf See Also}\\
\texttt{getJointState()}\\
%%%%% END OF ChNXT::connect %%%%%
