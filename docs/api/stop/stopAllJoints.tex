\noindent
\vspace{5pt}
\rule{4.5in}{0.015in}\\
\noindent
{\LARGE \texttt{ChNXT::stopAllJoints()} \index{ChNXT::stopAllJoints()}}\\

\addcontentsline{toc}{subsection}{stopAllJoints()}

\noindent
{\bf Synopsis}
\begin{lstlisting}
#include <nxt.h>
int ChNXT::stopAllJoints(void);
\end{lstlisting}

\noindent
{\bf Purpose}\\
Stop all joints moving.\\

\noindent
{\bf Return Value}\\
The function returns 0 on success and non-zero otherwise.\\

\noindent
{\bf Parameters}\\
None.\\

\noindent
{\bf Description}\\
This function is used to stop all joints on the Mindstorm NXT.\\

\noindent
{\bf Example}
\begin{lstlisting}
#include <nxt.h> 

ChNXT nxt;

if (nxt.connectWithAddress("00:16:53:12:e7:80")){
    printf("Connection to NXT has failed.");
    exit(-1);
}
    
/* setup speed for all three joints */
nxt.setJointSpeeds(60, 40, 40);

/* move all joints forward */
nxt.moveJointContinuousNB(NXT_JOINTA, NXT_FORWARD);
nxt.moveJointContinuousNB(NXT_JOINTB, NXT_FORWARD);
nxt.moveJointContinuousNB(NXT_JOINTC, NXT_FORWARD);
delay(5);

/* stop all joints */
nxt.stopAllJoints();
\end{lstlisting}

\noindent
{\bf See Also}\\
\texttt{stopOneJoint(), stopTwoJoints()}\\
%%%%% END OF ChNXT::stopAllJoints %%%%%
