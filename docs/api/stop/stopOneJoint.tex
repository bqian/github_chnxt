\noindent
\vspace{5pt}
\rule{4.5in}{0.015in}\\
\noindent
{\LARGE \texttt{ChNXT::stopOneJoint()} \index{ChNXT::stopOneJoint()}}\\

\addcontentsline{toc}{subsection}{stopOneJoint()}

\noindent
{\bf Synopsis}
\begin{lstlisting}
#include <nxt.h>
int ChNXT::stopOneJoint(nxtJointId_t id);
\end{lstlisting}

\noindent
{\bf Purpose}\\
Stop a joint moving.\\

\noindent
{\bf Return Value}\\
The function returns 0 on success and non-zero otherwise.\\

\noindent
{\bf Parameters}\\
\vspace{-0.1in}
\begin{description}
\item
\begin{tabular}{ p{20mm}p{135mm} }
\texttt{id}       &The joint of the Mindstorm NXT.\\
\end{tabular}
\end{description}

\noindent
{\bf Description}\\
This function is used to stop the specific joint on the Mindstorm NXT.\\

\noindent
{\bf Example}
\begin{lstlisting}
#include <nxt.h> 

ChNXT nxt;

if (nxt.connectWithAddress("00:16:53:12:e7:80")){
    printf("Connection to NXT has failed.");
    exit(-1);
}
    
/* setup speed for all three joints */
nxt.setJointSpeeds(60, 40, 40);

/* move joint1 360 degrees */
nxt.moveJointContinuousNB(NXT_JOINTA, NXT_FORWARD);
delay(5);

/* stop joint1 */
nxt.stopOneJoint(NXT_JOINTA);
\end{lstlisting}

\noindent
{\bf See Also}\\
\texttt{stopTwoJoints(), stopAllJoints()}\\
%%%%% END OF ChNXT::connect %%%%%
